\documentclass[
    12pt, 
    usepdftitle=false,
    aspectratio=1610
]{beamer}

\usetheme{Madrid}

\usefonttheme[]{serif}
\setbeamertemplate{caption}[numbered]
\setbeamertemplate{navigation symbols}{}

\usepackage[T2A]{fontenc}			
\usepackage[utf8]{inputenc}			
\usepackage[english,russian]{babel}

\usepackage{
    mathtext,
    minted,
    cmap,
    multirow,
    textcomp,
    graphicx,
    wrapfig,
    subfig,
    mathtools,
    gensymb
}
\usepackage[font=small,labelfont=bf]{caption}

\graphicspath{{./figs/03}}

\title[Лекция 3]{Методы улучшения качества изображений}

\author{Александр Танченко}
\institute{}
\date{2025}

\begin{document}

%------------------------------------------------------------
\begin{frame}
    \titlepage
\end{frame}

%------------------------------------------------------------
\begin{frame}
\frametitle{Задача улучшения контрастности изображений}
\begin{figure}
    \centering
    \includegraphics[height=0.5\textheight]{low_contrast.png}
    \captionof*{figure}{Не контрастное vs. контрастное изображения}
\end{figure}
\begin{itemize}
    \item эквализация гистограммы изображения
    \item алгоритм CLAHE
\end{itemize}
\vspace{0.3cm}
\end{frame}


%------------------------------------------------------------
\begin{frame}
\frametitle{Гистограмма изображения}
\begin{figure}
    \centering
    \includegraphics[height=0.3\textheight]{hist_1.png}
\end{figure}
$\mathbf{A}$ -- одноканальное, восьмибитное изображение размера $H\times W$
    $$
        \mathbf{A}=\left\{a_{ij}\right\},\quad
        0\leqslant a_{ij}<L=256
    $$
    нормализованная гистограмма изображения
    $$
        p_k=\frac{n_k}{H\cdot W},\qquad
        k=0,1,\ldots L-1
    $$
    $n_k$ -- число пикселей изображения, имеющих яркость $k$
    $$
        p_k = \mathbf{P}[a_{ij}=k]\qquad
    $$
\end{frame}

%---------------------------------------------------------------------
\begin{frame}
    \frametitle{Примеры гистограмм}
    \begin{figure}[t]
        \includegraphics[width=0.9\textwidth]{histogram.pdf}
    \end{figure}
\end{frame}

%------------------------------------------------------------
\begin{frame}
\frametitle{Эквализация гистограммы изображения}
\begin{itemize}
        \item если у непрерывной случайной величины $X$ плотность распределения $p(x)$
        \item функция распределения
        $$
            y=F(x)=\int\limits_{-\infty}^xp(t)\,dt
        $$
        \item то случайная величина $Y=F\left(X\right)$ будет иметь равномерное распределение
        \item выравнивание гистограммы происходит по аналогичной формуле
        $$
            F_k=\sum_{i=0}^k p_i\qquad
            k=0,1,2,\ldots,L-1
        $$
        $$
            0\leqslant F_k \leqslant 1
        $$
        $$
            k\quad\mapsto\quad
            \left\lceil(L-1)\cdot F_k\right\rceil
        $$
    \end{itemize}
\end{frame}

\begin{frame}
    \frametitle{Пример эквализации гистограммы}
    \begin{center}
    \begin{tabular}{ |c|p{1cm}|c|c|c|  }
        \hline
        \multicolumn{5}{|c|}{\textbf{Трехбитовое изображение $L=8$, $64\times64$ пикселя}} \\
        \hline
        $k$ & $n_k$ & $p_k=n_k/64^2$ & $F_k=\sum_{i=0}^k p_i$ & $k\mapsto\lceil(L-1)\cdot F_k\rceil$\\
        \hline
        \hline
        0 & 790  & 0.19 & 0.19 & $0\mapsto1$\\
        1 & 1023 & 0.25 & 0.44 & $1\mapsto3$\\
        2 & 850  & 0.21 & 0.65 & $2\mapsto5$\\
        3 & 656  & 0.16 & 0.81 & $3\mapsto6$\\
        4 & 329  & 0.08 & 0.89 & $4\mapsto7$\\
        5 & 245  & 0.06 & 0.95 & $5\mapsto7$\\
        6 & 122  & 0.03 & 0.98 & $6\mapsto7$\\
        7 & 81   & 0.02 & 1.0  & $7\mapsto7$\\
        \hline
    \end{tabular}
    \end{center}
    \begin{figure}[t]%
        \centering
        \subfloat{
            \includegraphics[width=0.2\textwidth]{equalization1.pdf}
        }
        \subfloat{
            \includegraphics[width=0.2\textwidth]{equalization2.pdf}
        }
       
    \end{figure}
\end{frame}

%---------------------------------------------------------------------
\begin{frame}
    \frametitle{Примеры эквализации гистограмм изображений}
    \begin{figure}[t]
        \includegraphics[height=0.85\textheight]{equalization3.pdf}
    \end{figure}
\end{frame}

%------------------------------------------------------------
\begin{frame}
\frametitle{CLAHE (Contrast Limited Adaptive Histogram Equalization)}
\begin{figure}[t]
    \includegraphics[height=0.85\textheight]{CLAHE.png}
\end{figure}
\end{frame}

%------------------------------------------------------------
\begin{frame}
\frametitle{Задача повышения резкости изображения}
\begin{figure}[t]
    \includegraphics[height=0.5\textheight]{sharpening.png}
\end{figure}
\begin{itemize}
    \item повышение резкости с помощью оператора Лапласа
    \item повышение резкости с помощью нерезкого маскирования
\end{itemize}
\end{frame}

%------------------------------------------------------------
\begin{frame}
\frametitle{Повышение резкости с помощью оператора Лапласа}
\begin{figure}[t]
    \includegraphics[height=0.5\textheight]{unsharp3.png}
    \hspace{1cm}
    \includegraphics[height=0.5\textheight]{unsharp1.pdf}
\end{figure}
$$
    g(x,y)=f(x,y)-\Delta f(x,y)
$$
оператор Лапласа
$$
    \Delta f=\frac{\partial^2 f}{\partial x^2}+\frac{\partial^2 f}{\partial y^2}
$$
\end{frame}

%------------------------------------------------------------
\begin{frame}
\frametitle{Вычисление оператора Лапласа с помощью свертки}
\begin{itemize}
     \item девятиточечная разностная схема для оператора Лапласа
    \begin{align*} 
        \Delta f(x,y)\approx &f(x+1,y)+f(x-1,y)+f(x,y+1)+f(x,y-1) +\\
                            +&f(x-1,y-1)+f(x+1,y+1)+f(x-1,y+1)+f(x+1,y-1)\\
                            -&8f(x,y)
    \end{align*}
    \item фильтр Лапласа
    $$
        \Delta f\approx
        w \bullet f = w \circ f
    $$
    ядро свертки фильтра Лапласа
    $$
        w =
        \begin{bmatrix*}[r]
            1 & 1 & 1 \\
            1 & -8 & 1 \\
            1 & 1 & 1
        \end{bmatrix*}
    $$
\end{itemize}
\end{frame}

%------------------------------------------------------------
\begin{frame}
\frametitle{Пример повышения резкости с помощью фильтра Лапласа}
\begin{figure}[t]
    \includegraphics[height=0.6\textheight]{unsharp2.png}
\end{figure}
$$
    g = f -
    \begin{bmatrix*}[r]
        1 & 1 & 1 \\
        1 & -8 & 1 \\
        1 & 1 & 1
    \end{bmatrix*}
    \circ f
$$
\end{frame}

%---------------------------------------------------------------------
\begin{frame}
    \frametitle{Повышение резкости с помощью нерезкого маскирования}
    \begin{wrapfigure}{r}{0.4\textwidth}
        \begin{center}
            \includegraphics[width=0.3\textwidth]{unsharp1.pdf}
        \end{center}
    \end{wrapfigure}
    $$
        g_{\mbox{mask}}(x,y)=f(x,y)-f_{\mbox{blur}}(x,y)
    $$
    $$
        g(x, y) = f(x,y)+k\cdot g_{\mbox{mask}}(x,y)
    $$
    \begin{center}
        \includegraphics[width=0.6\textwidth]{unsharp2.pdf}
    \end{center}
    \begin{itemize}
        \item исходное изображение $f$
        \item результат гауссовской фильтрации $f_{\mbox{blur}}$
        \item $g_{\mbox{mask}}$
        \item маскирование при $k=1$ и $k=4.5$
    \end{itemize}
\end{frame}

%------------------------------------------------------------
\begin{frame}
\frametitle{Задача уменьшения шума на изображении}
\begin{center}
    \includegraphics[height=0.5\textheight]{noise_reduction.png}
\end{center}
\begin{itemize}
    \item медианная фильтрация
    \item гауссовская фильтрация
    \item билатеральная фильтрация
\end{itemize}
\end{frame}

%------------------------------------------------------------
\begin{frame}
\frametitle{Модели шума}
\begin{center}
    \includegraphics[height=0.5\textheight]{noises.png}
\end{center}
Аддитивный гауссовский шум и шум соли и перца
$$
    \mathbf{A}+\varepsilon
$$
\begin{itemize}
    \item шум $\varepsilon$ -- имеет нормальное распределение
    \item шум $\varepsilon$ -- имеет распределение Бернулли
\end{itemize}
\end{frame}

%------------------------------------------------------------
\begin{frame}
    \frametitle{Уменьшение шума с помощью медианной фильтрации}
    \begin{center}
        \includegraphics[width=0.4\textwidth]{median_filter.png}
    \end{center}
    \begin{itemize}
        \item нелинейный фильтр
        \item устойчив к выбросам (salt-and-pepper noise)
        \item не сильно размывает границы
    \end{itemize}
    \begin{center}
        \includegraphics[width=0.7\textwidth]{median_filter_sample.png}
    \end{center}
\end{frame}

%------------------------------------------------------------
\begin{frame}
\frametitle{Уменьшение шума с помощью гауссовской фильтрации}
\begin{center}
    \includegraphics[height=0.3\textheight]{gauss_kernel.pdf}
    \includegraphics[height=0.3\textheight]{gauss_filter.png}
\end{center}
Гауссовское ядро
$$
    w(s,t)=\exp\left\{-\frac{s^2+t^2}{2\sigma_s^2}\right\}
    \qquad n = 2k+1\qquad k=\lceil 3\sigma_s\rceil
$$
Гауссовский фильтр
$$
    g(x,y)=\frac{\sum\limits_{s=-k}^k\sum\limits_{t=-k}^k w(s,t)f(x+s,y+t)}{\sum\limits_{s=-k}^k\sum\limits_{t=-k}^kw(s,t)}=
    w\bullet f=w\circ f
$$
\end{frame}

%------------------------------------------------------------
\begin{frame}
\frametitle{Уменьшение шума с помощью билатеральной фильтрации}
\begin{center}
    \includegraphics[width=0.4\textwidth]{bilateral_filter.png}
\end{center}
$$
    w(s,t,x,y)=\exp\left\{-\frac{s^2+t^2}{2\sigma_s^2}-\frac{\left[f(x+s,y+t)-f(x,y)\right]^2}{2\sigma_c^2}\right\}
    \quad n = 2k+1\quad k=\lceil 3\sigma_s\rceil
$$
билатеральная фильтрация
$$
    g(x,y)=\frac{\sum\limits_{s=-k}^k\sum\limits_{t=-k}^k w(s,t,x,y)f(x+s,y+t)}{\sum\limits_{s=-k}^k\sum\limits_{t=-k}^kw(s,t,x,y)}
$$
\end{frame}

%------------------------------------------------------------
\end{document}

%------------------------------------------------------------
\begin{frame}
\frametitle{}
\end{frame}