\documentclass[
    12pt, 
    usepdftitle=false,
    aspectratio=1610
]{beamer}

\usetheme{Madrid}

\usefonttheme[]{serif}
\setbeamertemplate{caption}[numbered]
\setbeamertemplate{navigation symbols}{}

\usepackage[T2A]{fontenc}			
\usepackage[utf8]{inputenc}			
\usepackage[english,russian]{babel}

\usepackage{
    mathtext,
    minted,
    cmap,
    multirow,
    textcomp,
    graphicx,
    wrapfig,
    subfig,
    mathtools,
    gensymb
}
\usepackage[font=small,labelfont=bf]{caption}

\graphicspath{{./figs/01}}

\title[Лекция 1]{
    Цветовые и геометрические операции над изображениями
}

\author{Александр Танченко}
\institute{}
\date{2025}

\begin{document}

%------------------------------------------------------------
\begin{frame}
    \titlepage
\end{frame}


%------------------------------------------------------------
\begin{frame}
\frametitle{План лекции}
\begin{itemize}
    \item RGB изображение
    \item цветовые пространства YCrCb, Grayscale и HSV
    \item ColorJitter аугментация 
    \item евклидовы и однородные координаты на плоскости 
    \item геометрические преобразования над изображениями
    \begin{itemize}
        \item преобразование подобия
        \item аффинное преобразование
        \item проективное преобразование (гомография)
    \end{itemize}
    %\item гамма коррекция изображений
    %\item эквализация гистограммы изображения
\end{itemize}
\end{frame}


%------------------------------------------------------------
\begin{frame}
\frametitle{RGB изображение}
\begin{figure}
    \centering
    \includegraphics[height=0.35\textheight]{coffee.png}
    \includegraphics[height=0.35\textheight]{rgb_image.png}
\end{figure}
\begin{itemize}
    \item RGB изображение -- массив размера $H\times W\times 3$
    \item изображение разбивается на цветовые плоскости (каналы)
    $$
        \mathbf{X}=\mathbf{R}\cup\mathbf{G}\cup\mathbf{B}
    $$
    \item изображение состоит из пикселей
    $$
        \mathbf{X}=\bigcup\limits_{ij}\mathbf{p}_{ij}
        \qquad
        \mathbf{p}_{ij}=\left(r_{ij}, g_{ij}, b_{ij}\right)
        \qquad
        i=0,1,\ldots,H-1
        \quad
        j=0,1,\ldots,W-1
    $$
    \item $0\leqslant r_{ij}, g_{ij}, b_{ij} \leqslant 255$
\end{itemize}
\end{frame}

%------------------------------------------------------------
\begin{frame}
\frametitle{Цветовое пространсво YCrCb}
\begin{figure}
    \centering
    \includegraphics[height=0.3\textheight]{YCrCb.pdf}
\end{figure}
\begin{itemize}
    \item преобразование из RGB в YCrCb
    \begin{align*}
        \mathbf{Y}  &= 0.299\cdot \mathbf{R} + 0.587\cdot \mathbf{G} + 0.114\cdot \mathbf{B} \\
        \mathbf{Cr} &= 0.713 \cdot (\mathbf{R} - \mathbf{Y}) + 0.5 \\
        \mathbf{Cb} &= 0.564 \cdot (\mathbf{B} - \mathbf{Y}) + 0.5 \\
    \end{align*}
    \item $\mathbf{Y}$ -- яркость изображения 
    \item $\mathbf{Cr}$, $\mathbf{Cb}$ -- цветовые каналы
\end{itemize}
\end{frame}

%------------------------------------------------------------
\begin{frame}
\frametitle{Изображение в градациях серого (grayscale image)}
\begin{figure}
    \centering
    \includegraphics[height=0.4\textheight]{grayscale.pdf}
\end{figure}
\begin{itemize}
    \item \textbf{grayscale image} -- яркость изображения
    $$
        \mathbf{Y} = 0.299\cdot \mathbf{R} + 0.587\cdot \mathbf{G} + 0.114\cdot \mathbf{B}
    $$
    \item grayscale image -- одноканальный 8-битный массив размера $H\times W$
    $$
        \mathbf{Y}=\left\{p_{ij}\right\}\qquad
        p_{ij}\in\{0,1,\ldots,255\}
    $$
    \item при изменении $0\leqslant p_{ij}\leqslant 255$, яркость меняется от черного к белому
\end{itemize}
\end{frame}

%------------------------------------------------------------
\begin{frame}
\frametitle{Цветовое пространство HSV}
\begin{figure}
    \centering
    \includegraphics[height=0.5\textheight]{HSV_model.png}
\end{figure}
Цветовое пространство HSV моделирует восприятие цвета человеком.
\begin{itemize}
    \item $0^\circ \leqslant H < 360^\circ$  \textbf{тон (hue)}
    \item $0\leqslant S \leqslant 1$  \textbf{насыщенность (saturation)}
    \item $0\leqslant V \leqslant 1$  \textbf{"яркость" (value)}
\end{itemize}
\end{frame}

%------------------------------------------------------------
\begin{frame}
\frametitle{Преобразование из RGB в HSV}
\begin{itemize}
    \item V (value) -- яркость
    $$
        V = \max(R,G,B)\qquad
        \mbox{где}\qquad
        R=\frac{r}{255.0}\,,\quad
        G=\frac{g}{255.0}\,,\quad
        B=\frac{b}{255.0}
    $$
    \item S (saturation) -- насыщенность
    $$
        S=1-\frac{\min(R,G,B)}{\max(R,G,B)}
    $$
    \item H (hue) -- тон
    $$
    H' = \frac{60}{V(1-S)}\cdot
    \begin{cases}
        G-B\,, & R=V\\
        B-R+2V(1-S)\,, & G=V\\
        R-G+4V(1-S)\,, & B=V
    \end{cases}
    $$
    $$
        H = H' \mod 360^\circ
    $$
\end{itemize}
\end{frame}

%------------------------------------------------------------
\begin{frame}
\frametitle{Преобразование из RGB в HSV}
\begin{figure}
    \centering
    \includegraphics[height=0.8\textheight]{HSV_sample.png}
\end{figure}
\end{frame}

%------------------------------------------------------------
\begin{frame}
\frametitle{Аугментация изображений}
\begin{figure}
    \centering
    \includegraphics[height=0.4\textheight]{augmentation.png}
\end{figure}
\begin{itemize}
    \item аугментация применяется, когда данных недостаточно для обучения нейронной сети или для повышения ее обобщающей способности
    \item два основных вида аугментации: цветовая, геометрическая
\end{itemize}
\end{frame}

%------------------------------------------------------------
\begin{frame}
\frametitle{Изменение тона}
\begin{figure}
    \centering
    \includegraphics[height=0.4\textheight]{hue_aug.png}
\end{figure}
\begin{itemize}
    \item конвертировать $\mathbf{RGB}\mapsto\mathbf{HSV}$
    \item изменить тон \textbf{H} (hue)
    $$
      \mathbf{H} = (\mathbf{H} + \alpha) \mod 360\,,\quad
      0 < \alpha < 360
    $$
    \item конвертировать $\mathbf{HSV}\mapsto\mathbf{RGB}$
\end{itemize}
\end{frame}

%------------------------------------------------------------
\begin{frame}
\frametitle{Изменение насыщенности}
\begin{figure}
    \centering
    \includegraphics[height=0.4\textheight]{sat_aug.png}
\end{figure}
\begin{itemize}
    \item найти grayscale изображение $\mathbf{Y}$
    \item преобразовать цветовые плоскости \textbf{R}, \textbf{G} и \textbf{B} по формулам
    \begin{align*}
      \mathbf{R} &= \beta \cdot \mathbf{R} + (1-\beta)\cdot\mathbf{Y}\\
      \mathbf{G} &= \beta \cdot \mathbf{G} + (1-\beta)\cdot\mathbf{Y}\\
      \mathbf{B} &= \beta \cdot \mathbf{B} + (1-\beta)\cdot\mathbf{Y}
    \end{align*}
  где $0 < \beta < 1$
\end{itemize}
\end{frame}

%------------------------------------------------------------
\begin{frame}
\frametitle{Изменение контрастности}
\begin{figure}
    \centering
    \includegraphics[height=0.4\textheight]{cont_aug.png}
\end{figure}
\begin{itemize}
    \item найти grayscale изображение $\mathbf{Y}$ и вычислить среднюю яркость $\mu\left(\mathbf{Y}\right)$
    \item преобразовать \textbf{RGB} изображение \textbf{A}
    $$
      \mathbf{A} = \gamma \cdot \mathbf{A} + (1-\gamma)\cdot\mu\left(\mathbf{A}\right)
    $$
  где $0 < \gamma < 1$
\end{itemize}
\end{frame}

%------------------------------------------------------------
\begin{frame}
\frametitle{Изменение яркости}
\begin{figure}
    \centering
    \includegraphics[height=0.4\textheight]{bright_aug.png}
\end{figure}
    \textbf{RGB} изображение \textbf{A} умножить на коэффициент $0 < \delta < 1$
    $$
        \mathbf{A} = \delta \cdot \mathbf{A}
    $$
\end{frame}

%------------------------------------------------------------
\begin{frame}
\frametitle{ColorJitter аугментация}
К изображению последовательно применяется
цветовые преобразования

\begin{itemize}
    \item изменения тона с параметром $\alpha\in[0,360]$
    \item изменения насыщенности с параметром $\beta\in[0,1]$
    \item изменения контрастности с параметром $\gamma\in[0,1]$
    \item изменения яркости с параметром $\delta\in[0,1]$
\end{itemize}

\end{frame}

%------------------------------------------------------------
\begin{frame}
\frametitle{Евклидовы (декартовы) координаты на плоскости}
    \begin{figure}
        \centering
        \includegraphics[height=0.4\textheight]{xyaxis.jpg}
    \end{figure}
    \begin{itemize}
        \item евклидовы координаты точек на плоскости
        $$
            \widetilde{\mathbf{x}}=
            \begin{bmatrix*}[c]
                x \\ y   
            \end{bmatrix*}\in\mathbf{R}^2
        $$
        \item для координат пикселей на изображении $x$, $y$ -- целые и
        \begin{align*}
            &0\leqslant x \leqslant W-1 \\
            &0\leqslant y \leqslant H-1
        \end{align*}
    \end{itemize}
\end{frame}

%------------------------------------------------------------
\begin{frame}
\frametitle{Однородные координаты на плоскости}
\begin{itemize}
    \item переход от евклидовых координат к однородным координатам на проективной плоскости $\mathbf{P}^2$
    $$
        \mathbf{R}^2\ni\widetilde{\mathbf{x}}=
        \begin{bmatrix*}[c]
            x \\ y   
        \end{bmatrix*}
        \quad\mapsto\quad
        \begin{bmatrix*}[c]
            x \\ y \\ 1
        \end{bmatrix*}
        =\mathbf{x}
        \in\mathbf{P}^2
    $$
    \item переход от однородных координат к  евклидовым
    $$
        \mathbf{P}^2\ni\mathbf{x}=
        \begin{bmatrix*}[c]
            x_1 \\ x_2 \\ x_3
        \end{bmatrix*}
        \quad\mapsto\quad
        \begin{bmatrix*}[c]
            x_1/x_3 \\ x_2/x_3
        \end{bmatrix*}
        =\widetilde{\mathbf{x}}
        \in\mathbf{R}^2
    $$
    \item $\mathbf{x}$ и $\lambda\cdot\mathbf{x}$ -- задают одну евклидову точку
\end{itemize}
\end{frame}

%---------------------------------------------------------------------
\begin{frame}
    \frametitle{Сдвиг системы координат}
    \begin{figure}[t]
        \includegraphics[width=0.6\textwidth]{shift.png}
    \end{figure}
    $$
        x' = x + t_x\,,\qquad
        y' = y + t_y
    $$
    $$
        \begin{bmatrix*}[c]
            x' \\ y' \\ 1    
        \end{bmatrix*}=
        \mathbf{T}(t_x,t_y)
        \begin{bmatrix*}[c]
            x \\ y \\ 1    
        \end{bmatrix*}
        \qquad
        \mathbf{T}(t_x,t_y)=
        \begin{bmatrix*}[r]
            1 & 0 & t_x \\
            0 & 1 & t_y \\
            0 & 0 & 1
        \end{bmatrix*}
    $$
    $$
        \begin{bmatrix*}[c]
            x \\ y \\ 1    
        \end{bmatrix*}=
        \mathbf{T}(-t_x,-t_y)
        \begin{bmatrix*}[c]
            x' \\ y' \\ 1    
        \end{bmatrix*}
    $$
\end{frame}

%---------------------------------------------------------------------
\begin{frame}
    \frametitle{Изменение масштаба системы координат}
    \begin{figure}[t]
        \includegraphics[width=0.6\textwidth]{scale.png}
    \end{figure}
    $$
        x' = f_x\cdot x\,,\qquad
        y' = f_y\cdot y
    $$
    $$
        \begin{bmatrix*}[c]
            x' \\ y' \\ 1    
        \end{bmatrix*}=
        \mathbf{F}(f_x,f_y)
        \begin{bmatrix*}[c]
            x \\ y \\ 1    
        \end{bmatrix*}
        \qquad
        \mathbf{F}(f_x,f_y)=
        \begin{bmatrix*}[r]
            f_x & 0 & 0 \\
            0 & f_y & 0 \\
            0 & 0 & 1
        \end{bmatrix*}
    $$
    $$
        \begin{bmatrix*}[c]
            x \\ y \\ 1    
        \end{bmatrix*}=
        \mathbf{F}(1/f_x,1/f_y)
        \begin{bmatrix*}[c]
            x' \\ y' \\ 1    
        \end{bmatrix*}
    $$
\end{frame}

%---------------------------------------------------------------------
\begin{frame}
    \frametitle{Поворот системы координат}
    \begin{figure}[t]
        \includegraphics[width=0.6\textwidth]{rotation.png}
    \end{figure}
    $$
        x' = x\cdot\cos\theta - y\cdot\sin\theta\,,\qquad
        y' = x\cdot\sin\theta + y\cdot\cos\theta
    $$
    $$
        \begin{bmatrix*}[c]
            x' \\ y' \\ 1    
        \end{bmatrix*}=
        \mathbf{R}(\theta)
        \begin{bmatrix*}[c]
            x \\ y \\ 1    
        \end{bmatrix*}
        \qquad
        \mathbf{R}(\theta)=
        \begin{bmatrix*}[r]
            \cos\theta & -\sin\theta & 0 \\
            \sin\theta & \cos\theta & 0 \\
            0 & 0 & 1
        \end{bmatrix*}
    $$
    $$
        \begin{bmatrix*}[c]
            x \\ y \\ 1    
        \end{bmatrix*}=
        \mathbf{R}(-\theta)
        \begin{bmatrix*}[c]
            x' \\ y' \\ 1    
        \end{bmatrix*}
    $$
\end{frame}

%---------------------------------------------------------------------
\begin{frame}
    \frametitle{Преобразование подобия (similarity transform)}
    \begin{figure}[t]
        \includegraphics[width=0.6\textwidth]{similarity.png}
    \end{figure}
    Преобразование подобия - поворот и масштабирование вокруг центра поворота
    $$
        \begin{bmatrix*}[c]
            x' \\ y' \\ 1    
        \end{bmatrix*}=
        \mathbf{S}\cdot
        \begin{bmatrix*}[c]
            x \\ y \\ 1    
        \end{bmatrix*}\qquad
        \mathbf{S}=
        \mathbf{T}(c_x,c_y)\cdot
        \mathbf{F}(f, f)\cdot
        \mathbf{R}(\theta)\cdot
        \mathbf{T}(-c_x,-c_y)
    $$
    $$
        \begin{bmatrix*}[c]
            x \\ y \\ 1    
        \end{bmatrix*}=
        \mathbf{S}^{-1}\cdot
        \begin{bmatrix*}[c]
            x' \\ y' \\ 1    
        \end{bmatrix*}
        \qquad
        \mathrm{dof}\left(\mathbf{S}\right)=4
    $$
\end{frame}

%---------------------------------------------------------------------
\begin{frame}
    \frametitle{Аффинное преобразование системы координат}
    \begin{figure}[t]
        \includegraphics[width=0.6\textwidth]{affine.png}
    \end{figure}
    $$
        \begin{bmatrix*}[c]
            x' \\ y' \\ 1    
        \end{bmatrix*}=
        \mathbf{A}\cdot
        \begin{bmatrix*}[c]
            x \\ y \\ 1    
        \end{bmatrix*}
        \qquad
        \mathbf{A}=
        \begin{bmatrix*}[r]
            a_{11} & a_{12} & a_{13} \\
            a_{21} & a_{22} & a_{23} \\
            0 & 0 & 1
        \end{bmatrix*}
        \qquad
        \mathrm{det}(\mathbf{A})\ne0
        \qquad
        \mathrm{dof}(\mathbf{A})=6
    $$
    Преобразование подобия -- частный случай аффинного преобразования.

    $\mathbf{A}$ можно вычислить, задав координаты трех соответствующих точек 
    $$
        \begin{bmatrix*}[c]
            x'_i \\ y'_i \\ 1    
        \end{bmatrix*}=
        \mathbf{A}\cdot
        \begin{bmatrix*}[c]
            x_i \\ y_i \\ 1    
        \end{bmatrix*}
        \qquad
        i=1,2,3
    $$
\end{frame}

%------------------------------------------------------------
\begin{frame}
\frametitle{Проективное преобразование (гомография)}
$$
    \begin{bmatrix*}[c]
        x'_1 \\ x'_2 \\ x'_3    
    \end{bmatrix*}=
    \mathbf{H}\cdot
    \begin{bmatrix*}[c]
        x \\ y \\ 1    
    \end{bmatrix*}\qquad
    \mathbf{H}=
    \begin{bmatrix*}[r]
        h_{11} & h_{12} & h_{13} \\
        h_{21} & h_{22} & h_{23} \\
        h_{31} & h_{32} & h_{33}
    \end{bmatrix*}
     \qquad
    \mathrm{det}(\mathbf{H})\ne0
    \qquad
    \mathrm{dof}(\mathbf{H})=8
$$
$$
    x'=\frac{x'_1}{x'_3}\qquad y'=\frac{x'_2}{x'_3}
$$
\begin{itemize}
    \item аффинное преобразование -- частный случай проективного
    \item матрицу $\mathbf{H}$ можно вычислить, задав координаты четырех соответствующих точек 
          на изображениях
    $$
        \begin{bmatrix*}[c]
            x'_i \\ y'_i 
        \end{bmatrix*}
        \quad\leftrightarrow\quad
        \begin{bmatrix*}[c]
            x_i \\ y_i   
        \end{bmatrix*}\qquad
        i=1,2,3,4
    $$
\end{itemize}
\end{frame}

%------------------------------------------------------------
\begin{frame}
\frametitle{Пимер проективного преобразования}
\begin{figure}[t]
    \includegraphics[width=0.5\textwidth]{projective3.pdf}
    \includegraphics[width=0.6\textwidth]{projective4.pdf}
\end{figure}
\end{frame}

%------------------------------------------------------------
\begin{frame}
\frametitle{Пимер проективного преобразования}
\begin{figure}[t]
    \includegraphics[width=0.4\textwidth]{projective1.pdf}
    \includegraphics[width=0.8\textwidth]{projective2.pdf}
\end{figure}
\end{frame}

%---------------------------------------------------------------------
\begin{frame}
    \frametitle{Интерполяция значений пикселей на изображении}
    \begin{figure}[t]
        \includegraphics[width=0.2\textwidth]{interpolation.png}
        \hspace{1cm}
        \includegraphics[width=0.5\textwidth]{projective5.pdf}
    \end{figure}
    $$
        \begin{bmatrix*}[c]
            x_1 \\ x_2 \\ x_3    
        \end{bmatrix*}=
        \mathbf{H}^{-1}
        \begin{bmatrix*}[c]
            x' \\ y' \\ 1    
        \end{bmatrix*}
    $$
    $$
        x=\frac{x_1}{x_3}\quad
        y=\frac{x_2}{x_3}
    $$
    $$
        \mathbf{p}'(x',y')=f(x,y) 
        \quad\mbox{-- зависит от метода интерполяции}
    $$
\end{frame}

%---------------------------------------------------------------------
\begin{frame}
    \frametitle{Интерполяция изображений методом ближайшего соседа}
    \begin{figure}[t]
        \includegraphics[width=0.2\textwidth]{interpolation.png}
    \end{figure}
    \begin{itemize}
        \item значения пикселей на преобразованном изображении равны
        $$
            \mathbf{p}'\left(x',y'\right)=f(x,y)
        $$
        \item в методе ближайшего соседа
        $$
            f(x,y)=\mathbf{p}\left(\mathrm{NearestNeighbor}(x,y)\right)
        $$
    \end{itemize}
\end{frame}

%---------------------------------------------------------------------
\begin{frame}
    \frametitle{Билинейная интерполяция изображений}
    \begin{figure}[t]
        \includegraphics[width=0.2\textwidth]{interpolation.png}
    \end{figure}
    \begin{itemize}
        \item значения пикселей на преобразованном изображении равны
        $$
            \mathbf{p}'\left(x',y'\right)=f(x,y)
        $$
        \item для билинейной интерполяции
        $$
            f(x,y)=a+b\cdot x+c\cdot y+d\cdot xy
        $$
        \item коэффициенты $a$, $b$, $c$, $d$ находятся из условий
        $$
            f(x_i,y_i)=\mathbf{p}(x_i, y_i)\qquad i=1,2,3,4
        $$
        \url{https://en.wikipedia.org/wiki/Bilinear_interpolation}
    \end{itemize}
\end{frame}

%---------------------------------------------------------------------
\begin{frame}
    \frametitle{Бикубическая интерполяция изображений}
    \begin{figure}[t]
        \includegraphics[width=0.2\textwidth]{interpolation3.png}
    \end{figure}
    \begin{itemize}
        \item значения пикселей на преобразованном изображении равны
        $$
            \mathbf{p}'\left(x',y'\right)=f(x,y)
        $$
        \item для билинейной интерполяции
        $$
            f(x,y)=\sum_{n,m=0}^3 f_{nm}x^ny^m
        $$
        \item коэффициенты $f_{nm}$ находятся из условий
        $$
            f(x_i,y_i)=\mathbf{p}(x_i, y_i)\qquad i=1,2,\ldots,16
        $$
    \end{itemize}
\end{frame}

%---------------------------------------------------------------------
\begin{frame}
    \frametitle{Сравнение интерполяций}
    \begin{figure}[t]
        \includegraphics[width=0.2\textwidth]{squares.png}
    \end{figure}
    \begin{figure}[t]
        \includegraphics[width=0.2\textwidth]{nn_interpol.png}
        \includegraphics[width=0.2\textwidth]{lin_interpol.png}
        \includegraphics[width=0.2\textwidth]{bilin_interpol.png}
        \caption*{Интерполяция методом ближайшего соседа, билинейная и бикубическая.}
    \end{figure}

\end{frame}

%---------------------------------------------------------------------
% \begin{frame}
% \frametitle{Гамма коррекция изображений}
% \begin{figure}
%     \centering
%     \includegraphics[height=0.35\textheight]{gamma.pdf}
%     \caption*{Графики функций $y=x^\gamma$, $0\leqslant x\leqslant 1$}
% \end{figure}
% \begin{itemize}
%     \item гамма коррекция изображения $\mathbf{A}=\left\{a_{ij}\right\}$
%     $$
%         \mathbf{B} = \left\{255\cdot\left(\frac{a_{ij}}{255}\right)^\gamma\right\}
%     $$
%     \item $\gamma<1$ изображение становиться светлее 
%     \item $\gamma>1$ изображение становиться темней
%     \item $\gamma=1$ изображение не меняется
% \end{itemize}
% \end{frame}

%---------------------------------------------------------------------
% \begin{frame}
% \frametitle{LUT-преобразование (look-up table)}
% \begin{itemize}
%     \item LUT таблица
%     $$
%         \mathrm{table}=\left\{t_0,t_1,\ldots,t_{255}\right\}
%     $$
%     \item индексом в этой таблице является значение входной интенсивности, а в соответствующей ячейке находится значение выходной интенсивности
%     \item LUT-преобразование сопоставляет каждому входному пикселю значение из таблицы 
%     $$
%         a_{ij} \quad\mapsto\quad t_{a_{ij}}
%     $$
%     \item время применения LUT к изображению значительно сокращается (обычно в изображениях
%     пикселей гораздо больше, чем значений интенсивности)
% \end{itemize}
% \end{frame}

%---------------------------------------------------------------------
% \begin{frame}
% \frametitle{Гистограмма изображения}
% \begin{figure}
%     \centering
%     \includegraphics[height=0.3\textheight]{cameraman.png}
%     \includegraphics[height=0.3\textheight]{histogram.png}
% \end{figure}
% \begin{itemize}
%     \item динамический диапазон пикселей $[0, 255]$ разбивается на интервалы (bins) 
%     $$
%         0=x_1<x_2<\cdots<x_{N+1}=255
%     $$
%     \item гистограмма изображения $\mathbf{A}=\left\{a_{ij}\right\}$ размера $H\times W$
%     $$
%         n_k = \left|\left\{x_k\leqslant a_{ij}< x_{k+1}\right\}\right|
%         \qquad
%         k=1,2,\ldots N
%     $$
%     \item нормализованная гистограмма
%     $$
%         p_k=\frac{n_k}{H\cdot W}\qquad
%         p_k=\mathbf{P}[x_k\leqslant a_{ij}< x_{k+1}]
%     $$
% \end{itemize}
% \end{frame}

%---------------------------------------------------------------------
% \begin{frame}
%     \frametitle{Прмеры гистограмм}
%     \begin{figure}[t]
%         \includegraphics[width=0.8\textwidth]{histogram.pdf}
%     \end{figure}
% \end{frame}

%---------------------------------------------------------------------
% \begin{frame}
%     \frametitle{Эквализация гистограммы изображения}
%     \begin{itemize}
%         \item если непрерывную случайную величину $X$ преобразовать с помощью ее функции распределения
%         $$
%             y=F(x)\qquad
%             F(x)=\mathbf{P}[X<x]=\int\limits_{-\infty}^xp_X(t)\,dt
%         $$
%         \item то  с.в. $Y=F(X)$ будет иметь равномерное распределение на $[0,1]$
%         \item по аналогии с непрерывной с.в., если преобразовать значения пикселей $a_{ij}$
%         $$
%            b_{ij}=\left\lceil255\cdot  \sum_{k=0}^{a_{ij}}p_k \right\rceil
%            \qquad \mbox{(вычисляется с помощью LUT)}
%         $$
%         получим примерно равномерное распределение пикселей $b_{ij}$
%     \end{itemize}
% \end{frame}

%--------------------------------------------------------
\end{document}
