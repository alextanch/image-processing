\documentclass[12pt, usepdftitle=false, aspectratio=1610]{beamer}

\usetheme{Madrid}
\usefonttheme[]{serif}
\setbeamertemplate{caption}[numbered]
\setbeamertemplate{navigation symbols}{}

\usepackage[T2A]{fontenc}			
\usepackage[utf8]{inputenc}			
\usepackage[english,russian]{babel}

\usepackage{
    mathtext,
    minted,
    cmap,
    multirow,
    textcomp,
    graphicx,
    wrapfig,
    subfig,
    mathtools,
    gensymb,
    amsmath,
    amsfonts,
    amssymb,
    hyperref,
    subfig,
    tikz, tikz-3dplot, xcolor, physics, bm
}
\usepackage[absolute, overlay]{textpos}
\usepackage[font=small,labelfont=bf]{caption}
\usepackage{subcaption}
\usetikzlibrary{calc,arrows.meta,positioning,backgrounds}

\DeclareMathOperator*{\argmax}{arg\,max}
\DeclareMathOperator*{\argmin}{arg\,min}

\graphicspath{{./figs/07}}

\title[Лекция 7]{Модель pinhole-камеры}

\author{Александр Танченко}
\institute{}
\date{2025}

\begin{document}

%------------------------------------------------------------
\begin{frame}
\frametitle{Трехмерное компьютерное зрение}
{\Large\bf Основные приложения 3D CV}
\vspace{0.5cm}
\begin{itemize}
    \item \textbf{SLAM (Simultaneous Localization and Mapping)} --- по набору изображений строится карта местности и ищется положение камеры.
    \vspace{0.2cm}
    
    \href{https://www.youtube.com/watch?v=GI90GYOCYuE}{{SLAM demo}}
    \vspace{0.2cm}

    \item \textbf{SfM (Structure from Motion)} --- реконструкция трехмерной сцены по набору изображений.
    \vspace{0.2cm}

    \href{https://www.youtube.com/shorts/0yVtPQULpd8}{{SfM demo}}

\end{itemize}
\end{frame}


%------------------------------------------------------------
\begin{frame}
\frametitle{Основы 3D компьютерного зрения}
{\Large\bf Содержание модуля}
\vspace{0.5cm}
\begin{itemize}
    \item Модель pinhole-камеры.
    \item Основные задачи 3D CV для трехмерных сцен.
    \item Двумерные преобразования изображений.
    \item Основные задачи 3D CV для сцен на плоскости.
    \item Эпиполярная геометрия.
    \item Реконструкция трехмерных сцен.
\end{itemize}
\end{frame}

%------------------------------------------------------------
\begin{frame}
\frametitle{Используемая литература}
    {\large\bf Основная:}
    \begin{itemize}
        \item Simon J.D. Prince
        <<Computer Vision: Models, Learning, and Inference>> 
        \url{https://udlbook.github.io/cvbook} \\
        \vskip0.2cm
        (Главы 14-16)
    \end{itemize}
    \vspace{0.5cm}
    {\large\bf Дополнительная:}
    \begin{itemize}
        \item R. Hartley, A. Zisserman 
        <<Multiple View Geometry in Computer Vision>>
        \vskip0.2cm
        \item Yi Ma, etc. 
        <<An Invitation to 3-D Vision: From Images to Geometric Models>> 
    \end{itemize}   
\end{frame}

%------------------------------------------------------------
\begin{frame}
\titlepage
\end{frame}

%------------------------------------------------------------
\begin{frame}
\frametitle{Однородные координаты на плоскости}
\begin{itemize}
    \item Преобразование декартовых координат на плоскости в однородные координаты
    $$
        \mathbf{x} =
        \begin{bmatrix}
            x \\
            y
        \end{bmatrix}
        \rightarrow
        \lambda\cdot
        \begin{bmatrix}
            x \\
            y \\
            1
        \end{bmatrix} =
        \widetilde{\mathbf{x}}
        \qquad
        \lambda \neq 0
    $$

    \item Это избыточное представление - разным $\lambda$ соответствуют одна декартовая точка.

    \item При преобразовании декартовых координат в однородные,
    обычно выбирают $\lambda = 1$
    $$
        \mathbf{x} =
        \begin{bmatrix}
            x \\
            y
        \end{bmatrix}
        \rightarrow
        \begin{bmatrix}
            x \\
            y \\
            1
        \end{bmatrix} =
        \widetilde{\mathbf{x}}
    $$
\end{itemize}
\end{frame}

%------------------------------------------------------------
\begin{frame}
\frametitle{Однородные координаты на плоскости}
\begin{itemize}
    \item Преобразование из однородных координат в декартовы:
    $$
        \widetilde{\mathbf{x}} =
        \begin{bmatrix}
            \tilde{x} \\
            \tilde{y} \\
            \tilde{z}
        \end{bmatrix}
        \rightarrow
        \begin{bmatrix}
            \tilde{x}/\tilde{z} \\
            \tilde{y}/\tilde{z}
        \end{bmatrix} =
        \mathbf{x}
        \qquad
        \widetilde{z} \neq 0
    $$
    
    \item Точки с $\widetilde{z} = 0$ называются точками на бесконечности.
    \item Пример:
    $$
        \widetilde{\mathbf{x}} =
        \begin{bmatrix}
            2 \\
            3 \\
            1
        \end{bmatrix} = 
        \begin{bmatrix}
            4 \\
            6 \\
            2
        \end{bmatrix}
        \rightarrow 
        \begin{bmatrix}
            4/2 \\
            6/2
        \end{bmatrix} =
        \begin{bmatrix}
            2 \\
            3
        \end{bmatrix} =
        \mathbf{x}
    $$
\end{itemize}
\end{frame}

%------------------------------------------------------------
\begin{frame}
\frametitle{Однородные координаты в пространстве}
\begin{itemize}
    \item Преобразование трехмерных декартовых координат в однородные координаты:
    $$
        \mathbf{w} =
        \begin{bmatrix}
            u \\
            v \\
            w
        \end{bmatrix}
        \rightarrow
        \lambda\cdot
        \begin{bmatrix}
            u \\
            v \\
            w \\
            1
        \end{bmatrix} = 
        \widetilde{\mathbf{w}}
        \qquad \lambda \neq 0
    $$
    \item Обратные преобразование из однородных координат в декартовы:
    $$
        \widetilde{\mathbf{w}}=
        \begin{bmatrix}
            \tilde{u} \\
            \tilde{v} \\
            \tilde{w} \\
            \tilde{z}
        \end{bmatrix}
        \rightarrow
        \begin{bmatrix}
            \tilde{u}/\tilde{z} \\
            \tilde{v}/\tilde{z} \\
            \tilde{w}/\tilde{z} \\
        \end{bmatrix}=
        \mathbf{w}
    $$
    \item Точки с $\tilde{z} = 0$ называются точками на бесконечности.           
\end{itemize}
\end{frame}

%------------------------------------------------------------
\begin{frame}
\frametitle{Преобразования трехмерных декартовых координат}

\begin{center}
\tdplotsetmaincoords{-60}{-35}
\begin{tikzpicture}[
        tdplot_main_coords,
        scale=0.9,
        >=Stealth,
        scale=0.8,
    ]   
    \coordinate (O) at (0,0,0);

    \draw [thick, ->, every node/.style={font=\footnotesize, inner sep=1pt}] 
        (O)     edge node [pos=1, anchor=north east] {$w_2$} % 2
        +(0,1.5,0) edge node [pos=1, anchor=north east] {$v_2$} % 3
        +(0,0,1.5) -- 
        +(1.5,0,0) node [anchor=north west] {$u_2$} % 1
    ;

    \tdplotsetrotatedcoords{-10}{40}{30}
    \coordinate (T) at (2.5,-2,-3);
    \tdplotsetrotatedcoordsorigin{(T)}
    
    \draw [tdplot_rotated_coords, thick, ->, every node/.style={font=\footnotesize, inner sep=1pt}] 
        (0,0,0) edge node [pos=1, anchor=north east] {$w_1$} % 2 -> z
        +(0,1.5,0) edge node [pos=1, anchor=north east] {$v_1$} % 3 -> y
        +(0,0,1.5) -- 
        +(1.5,0,0) node [anchor=north west] {$u_1$} % 1 -> x
    ;
    \coordinate (P) at (-2,3,-2);
    
    \path [every node/.style={font=\footnotesize, inner sep=4pt}] 
        (P) node [above right] {$\mathbf{w_2}=[u_2,v_2,w_2]$};

    \path [every node/.style={font=\footnotesize, inner sep=4pt}] 
        (P) node [below right] {$\mathbf{w_1}=[u_1,v_1,w_1]$};

    % \path [every node/.style={font=\footnotesize, inner sep=2pt}] 
    %     ($0.5*(O)+0.5*(T)$) node [below right] {$\mathbf{t}$};

    % \path [every node/.style={font=\footnotesize, inner sep=2pt}] 
    %     (O) node [below right] {$\mathbf{o}_2$};

    % \path [every node/.style={font=\footnotesize, inner sep=2pt}] 
    %     (T) node [right] {$\mathbf{o}_1$};

    %\draw [thick, blue, dashed] (O.center) -- (T.center);
    \draw [thick, black, fill=blue] (O) circle [radius=1.2pt];
    \draw [thick, black, fill=blue] (T) circle [radius=1.2pt];
    \draw [thick, black, fill=red] (P) circle [radius=2pt];
\end{tikzpicture}
\end{center}

\begin{itemize}
    \item Координаты точки $\mathbf{w}_1$ в первой системе координат связаны с координатами
    точки $\mathbf{w}_2$ во второй системе следующим образом
    $$
        \mathbf{w}_1 = \mathbf{R}\cdot\mathbf{w}_2 + \mathbf{t}
    $$
    \item где
        \begin{align*}
            \mathbf{R} &\quad \mbox{-- матрица вращения} \qquad \left(\mathbf{R}^\top=\mathbf{R}^{-1}\,, \quad \det\mathbf{R}=1\right)\\
            \mathbf{t} &\quad \mbox{-- вектор сдвига}
        \end{align*}
\end{itemize}
\end{frame}

%------------------------------------------------------------
\begin{frame}
\frametitle{Преобразования трехмерных декартовых координат}

\begin{center}
\tdplotsetmaincoords{-60}{-35}
\begin{tikzpicture}[
        tdplot_main_coords,
        scale=0.9,
        >=Stealth,
        scale=0.8,
    ]   
    \coordinate (O) at (0,0,0);

    \draw [thick, ->, every node/.style={font=\footnotesize, inner sep=1pt}] 
        (O)     edge node [pos=1, anchor=north east] {$w_2$} % 2
        +(0,1.5,0) edge node [pos=1, anchor=north east] {$v_2$} % 3
        +(0,0,1.5) -- 
        +(1.5,0,0) node [anchor=north west] {$u_2$} % 1
    ;

    \tdplotsetrotatedcoords{-10}{40}{30}
    \coordinate (T) at (2.5,-2,-3);
    \tdplotsetrotatedcoordsorigin{(T)}
    
    \draw [tdplot_rotated_coords, thick, ->, every node/.style={font=\footnotesize, inner sep=1pt}] 
        (0,0,0) edge node [pos=1, anchor=north east] {$w_1$} % 2 -> z
        +(0,1.5,0) edge node [pos=1, anchor=north east] {$v_1$} % 3 -> y
        +(0,0,1.5) -- 
        +(1.5,0,0) node [anchor=north west] {$u_1$} % 1 -> x
    ;
    \coordinate (P) at (-2,3,-2);
    
    % \path [every node/.style={font=\footnotesize, inner sep=4pt}] 
    %     (P) node [above right] {$\mathbf{w_2}=[u_2,v_2,w_2]$};

    % \path [every node/.style={font=\footnotesize, inner sep=4pt}] 
    %     (P) node [below right] {$\mathbf{w_1}=[u_1,v_1,w_1]$};

    \path [every node/.style={font=\footnotesize, inner sep=2pt}] 
        ($0.5*(O)+0.5*(T)$) node [below right] {$\mathbf{t}$};

    \path [every node/.style={font=\footnotesize, inner sep=2pt}] 
        (O) node [below right] {$\mathbf{o}_2$};

    \path [every node/.style={font=\footnotesize, inner sep=2pt}] 
        (T) node [right] {$\mathbf{o}_1$};

    \draw [thick, blue, dashed] (O.center) -- (T.center);
    \draw [thick, black, fill=blue] (O) circle [radius=1.2pt];
    \draw [thick, black, fill=blue] (T) circle [radius=1.2pt];
    % \draw [thick, black, fill=red] (P) circle [radius=2pt];
\end{tikzpicture}
\end{center}

$$
    \mathbf{w}_1 = \mathbf{R}\cdot\mathbf{w}_2 + \mathbf{t}
$$
Координаты точки $\mathbf{o}_2$ во второй и первой системе координат:
$$
    \mathbf{w}_2 = \mathbf{0}
    \qquad
    \mathbf{w}_1 = \mathbf{t}
$$
Значит вектор сдвига $\mathbf{t}$ --- это координаты начала второй системы координат в первой системе координат.
\end{frame}

%------------------------------------------------------------
\begin{frame}
\frametitle{Преобразования трехмерных декартовых координат}

\begin{center}
\tdplotsetmaincoords{-60}{-35}
\begin{tikzpicture}[
        tdplot_main_coords,
        >=Stealth,
        scale=0.7,
    ]   
    \coordinate (O) at (0,0,0);

    \draw [thick, ->, every node/.style={font=\footnotesize, inner sep=1pt}] 
        (O)     edge node [pos=1, anchor=north east] {$w_2$} % 2
        +(0,2,0) edge node [pos=1, anchor=north east] {$v_2$} % 3
        +(0,0,2) -- 
        +(2,0,0) node [anchor=north west] {$u_2$} % 1
    ;

    \draw [blue, thick, ->, every node/.style={font=\footnotesize, inner sep=1pt}] 
        (0,0,0) edge node [pos=1, anchor=south west] {$\mathbf{e}_3$} % 2 -> z
        +(0,1,0) edge node [pos=1, anchor=north east] {$\mathbf{e}_2$} % 3 -> y
        +(0,0,1) -- 
        +(1,0,0) node [anchor=north west] {$\mathbf{e}_1$} % 1 -> x
    ;
    \tdplotsetrotatedcoords{-10}{40}{30}
    \coordinate (T) at (2.5,-2,-3);
    \tdplotsetrotatedcoordsorigin{(T)}
    
    \draw [tdplot_rotated_coords, thick, ->, every node/.style={font=\footnotesize, inner sep=1pt}] 
        (0,0,0) edge node [pos=1, anchor=north east] {$w_1$} % 2 -> z
        +(0,2,0) edge node [pos=1, anchor=north east] {$v_1$} % 3 -> y
        +(0,0,2) -- 
        +(2,0,0) node [anchor=north west] {$u_1$} % 1 -> x
    ;
    \coordinate (P) at (-2,3,-2);

    % \draw [tdplot_rotated_coords, blue, thick, ->, every node/.style={font=\footnotesize, inner sep=1pt}] 
    %     (0,0,0) edge node [pos=1, anchor=south west] {$\mathbf{e}_3$} % 2 -> z
    %     +(0,1,0) edge node [pos=1, anchor=north east] {$\mathbf{e}_2$} % 3 -> y
    %     +(0,0,1) -- 
    %     +(1,0,0) node [anchor=north west] {$\mathbf{e}_1$} % 1 -> x
    % ;

    %\draw [thick, blue, dashed] (O.center) -- (T.center);
    %\draw [thick, black, fill=blue] (O) circle [radius=1.2pt];
    %\draw [thick, black, fill=blue] (T) circle [radius=1.2pt];
    % \draw [thick, black, fill=red] (P) circle [radius=2pt];
\end{tikzpicture}
\end{center}
$$
    \mathbf{w}_1 = \mathbf{R}\cdot\mathbf{w}_2 + \mathbf{t}
$$
Столбцы матрицы вращения $\mathbf{R}$ --- это координаты ортов второй системы координат в первой системе координат:
$$
    \mathbf{R} =
    \begin{bmatrix}
            r_{11} & r_{12} & r_{13} \\
            r_{21} & r_{22} & r_{23} \\
            r_{31} & r_{32} & r_{33}    
    \end{bmatrix}\quad
    \mathbf{e}_1 = \begin{bmatrix} r_{11} \\ r_{21} \\ r_{31} \end{bmatrix},\quad
    \mathbf{e}_2 = \begin{bmatrix} r_{12} \\ r_{22} \\ r_{32} \end{bmatrix},\quad
    \mathbf{e}_3 = \begin{bmatrix} r_{13} \\ r_{23} \\ r_{33} \end{bmatrix}
$$
\end{frame}

%------------------------------------------------------------
\begin{frame}
\frametitle{Преобразования однородных координат }
    Преобразование декартовых координат при вращении и сдвиге:
    $$
        \mathbf{w}_1 = \mathbf{R}\cdot\mathbf{w}_2 + \mathbf{t}
    $$
    в однородных координатах
    $$
        \widetilde{\mathbf{w}}_2 =
        \begin{bmatrix}
            \mathbf{w}_2 \\
            1
        \end{bmatrix}=
        \begin{bmatrix}
            u_2 \\ v_2 \\ w_2 \\ 1
        \end{bmatrix}\,,
        \qquad
        \widetilde{\mathbf{w}}_1 =
        \begin{bmatrix}
            \mathbf{w}_1 \\
            1
        \end{bmatrix}=
        \begin{bmatrix}
            u_1 \\ v_1 \\ w_1 \\ 1
        \end{bmatrix}
    $$
    записывается следующим образом:
    $$
        \widetilde{\mathbf{w}}_1 =
        \begin{bmatrix}
            \mathbf{R} & \mathbf{t} \\
            \mathbf{0} & 1
        \end{bmatrix}
        \cdot
        \widetilde{\mathbf{w}}_2
    $$
\end{frame}

%------------------------------------------------------------
\begin{frame}
\frametitle{Пример преобразования однородных координат}
    При вращении и сдвиге однородные координаты преобразуются следующим образом:
    $$
        \mathbf{w}_1 = 
        \begin{bmatrix*}[r]
            0 & 0 & -1 \\
            1 & 0 & 0 \\
            0 & -1 & 0 
        \end{bmatrix*}
        \cdot
        \mathbf{w}_2  +
        \begin{bmatrix}
            6 \\
            0 \\
            5   
        \end{bmatrix}
    $$
    Выразить координаты $\mathbf{w}_2$ через $\mathbf{w}_1$ и
    нарисовать системы координат.
\end{frame}

%------------------------------------------------------------
\begin{frame}
\frametitle{Параметризация матриц вращения}
У матрицы вращения $\mathbf{R}$ девять коэффициентов, но  только три степени свободы.
$$
    \mathbf{R} =
    \begin{bmatrix}
    r_{11} & r_{12} & r_{13} \\
    r_{21} & r_{22} & r_{23} \\
    r_{31} & r_{32} & r_{33}
    \end{bmatrix}
    \qquad
    \mathrm{dof}(\mathbf{R}) = 3
$$  

Существует несколько способов параметризации матриц вращения c тремя параметрами:
\begin{itemize}
    \item параметризация с помощью углов Эйлера
    \item параметризация с помощью кватернионов
    \item параметризация с помощью \textbf{вектора вращения}
\end{itemize}  
\end{frame}

%------------------------------------------------------------
\begin{frame}
\frametitle{Параметризация с помощью вектора вращения}
\begin{center}
\begin{tikzpicture}	[scale=0.4,>=Stealth]
    \draw[very thick,-latex,blue] (0,0) -- (6,0) node [above] {$v_2$};
    \draw[very thick,-latex,blue] (0,0) -- (0,6) node [left] {$w_2$};
    \draw[very thick,-latex,blue] (0,0) -- (-3,-3) node [left] {$u_2$};

    \draw[very thick,-latex] (0,0) -- (1.5,5.5) node [right] {$w_1$};
    \draw[very thick,-latex] (0,0) -- (5.5,-1.5) node [below] {$v_1$};
    \draw[very thick,-latex] (0,0) -- (-4,-1.5) node [above] {$u_1$};
    
    \draw[very thick,-latex, red] (0,0) -- (3,2) node [above] {$\boldsymbol{\omega}$};

    \filldraw[black] (0,0) circle (3pt) ;
\end{tikzpicture}
\end{center}
\begin{itemize}
    \item Любое вращение в трехмерном пространстве задается осью вращения
    (единичным вектором $\mathbf{n}$) и углом вращения $\theta$ вокруг этой оси.
    \item Вектор вращения $\boldsymbol{\omega} = \theta\cdot\mathbf{n}$ задает
    ось и угол вращения, а значит и матрицу вращения $\mathbf{R}$.
    \item $\mathrm{dof}(\boldsymbol{\omega}) = 3$
    \item Эта параметризация $\mathbf{R}$ неоднозначна, так как
    вращение на угол $\theta + 2\pi k$, $k \in \mathbb{Z}$, 
    вокруг оси $\mathbf{n}$ эквивалентно вращению на угол $\theta$ вокруг оси $\mathbf{n}$.
\end{itemize}
\end{frame}

%------------------------------------------------------------
\begin{frame}
\frametitle{Формулы Родрига}
\begin{itemize}
    \item Матрица вращения $\mathbf{R}$ может быть выражена через вектор вращения 
    $\boldsymbol{\omega} = \theta\cdot\mathbf{n}$
    с помощью формулы Родрига:
    $$
        \mathbf{R} = \mathbf{I} + \sin\theta\,[\mathbf{n}]_\times + (1-\cos\theta)\,[\mathbf{n}]_\times^2
    $$
    где $[\mathbf{n}]_\times$ --- кососимметрическая матрица, соответствующая вектору $\mathbf{n}$:
    $$
        [\mathbf{n}]_\times =
        \begin{bmatrix*}[c]
            0       & -n_3   & n_2 \\
            n_3     & 0      & -n_1 \\
            -n_2    & n_1    & 0
        \end{bmatrix*}
    $$
\end{itemize}
\end{frame}

%------------------------------------------------------------
\begin{frame}
\frametitle{Обратная формула Родрига}
Обратное преобразование из матрицы вращения 
$$
    \mathbf{R} =
    \begin{bmatrix}
    r_{11} & r_{12} & r_{13} \\
    r_{21} & r_{22} & r_{23} \\
    r_{31} & r_{32} & r_{33}
    \end{bmatrix}
$$  
в вектор вращения $\boldsymbol{\omega} = \theta\cdot\mathbf{n}$:
    $$
        |\boldsymbol{\omega}| = \arccos\left(\frac{\mathrm{trace}(\mathbf{R}) - 1}{2}\right)
    $$
    $$
        \boldsymbol{\omega} = 
        \frac{|\boldsymbol{\omega}|}{2\sin|\boldsymbol{\omega}|}
        \begin{bmatrix}
            r_{32} - r_{23} \\
            r_{13} - r_{31} \\
            r_{21} - r_{12}
        \end{bmatrix}
    $$
\end{frame}

%------------------------------------------------------------
\begin{frame}
\frametitle{Пример обратной формулы Родрига}
    Для матрицы вращения:
    $$
        \mathbf{R} =
        \begin{bmatrix}
            0 & -1 & 0 \\
            1 & 0  & 0 \\     
            0 & 0  & 1
        \end{bmatrix}
    $$
    Вычислим вектор вращения $\boldsymbol{\omega}$:
    $$
        \theta = 
        \arccos\left(\frac{\mathrm{trace}(\mathbf{R}) - 1}{2}\right) =
        \arccos\left(\frac{1 - 1}{2}\right) = \arccos(0) = \pi/2
    $$
    $$
        \mathbf{n} = 
        \frac{1}{2\sin\theta}
        \begin{bmatrix}
            r_{32} - r_{23} \\
            r_{13} - r_{31} \\
            r_{21} - r_{12}
        \end{bmatrix} =
        \frac{1}{2\sin(\pi/2)}\cdot
        \begin{bmatrix}
            0 - 0 \\
            0 - 0 \\
            1 - (-1)
        \end{bmatrix} =
        \begin{bmatrix}
            0 \\
            0 \\
            1
        \end{bmatrix}     
    $$
    $$
        \boldsymbol{\omega} = \theta \cdot \mathbf{n} = 
        \begin{bmatrix}
            0 \\
            0 \\
            \pi/2
        \end{bmatrix}  
        \quad\mbox{-- вращение на угол $90^\circ$ вокруг оси $w$.}
    $$
\end{frame}

%------------------------------------------------------------
\begin{frame}
\frametitle{Пример формулы Родрига}
\begin{itemize}
    \item Обратно, пусть теперь дан вектор вращения $\boldsymbol{\omega}=\left[0, 0, \pi/2\right]^\top$
    \item Найдем угол вращения $\theta$ и единичный вектор оси вращения $\mathbf{n}$:
    $$
        \theta = |\boldsymbol{\omega}| = \frac{\pi}{2}\,,
        \qquad
        \mathbf{n} = \frac{\boldsymbol{\omega}}{|\boldsymbol{\omega}|} =
        \begin{bmatrix} 
            0 \\ 0 \\ 1
        \end{bmatrix}
    $$
    \item Теперь найдем кососимметрическую матрицу $[\mathbf{n}]_\times$:
    $$
        [\mathbf{n}]_\times =
        \begin{bmatrix}
            0       & -n_3   & n_2 \\
            n_3     & 0      & -n_1 \\
            -n_2    & n_1    & 0
        \end{bmatrix} =
        \begin{bmatrix}
            0       & -1   & 0 \\
            1       & 0    & 0 \\
            0       & 0    & 0
        \end{bmatrix}
    $$
    \item Подставляя полученные значения в формулу Родрига, находим матрицу вращения:
    $$
        \mathbf{R} = \mathbf{I} + \sin\frac{\pi}{2}\,[\mathbf{n}]_\times + \left(1-\cos\frac{\pi}{2}\right)\,[\mathbf{n}]_\times^2 =
        \begin{bmatrix}
            0 & -1 & 0 \\
            1 & 0  & 0 \\     
            0 & 0  & 1
        \end{bmatrix}
    $$
    
\end{itemize}
\end{frame}

%------------------------------------------------------------
\begin{frame}
\frametitle{Pinhole-камера}
\begin{figure}[h]
    \centering
    \includegraphics[width=0.4\textwidth]{obscure.jpg}
    \hspace{0.5cm}
    \includegraphics[width=0.31\textwidth]{pinhole.jpg}
\end{figure}

\begin{itemize}
    \item \textbf{Pinhole-камера} (камера-обскура) --- простейшая модель камеры, 
    в которой изображение формируется на плоскости проекции 
    через маленькое отверстие.
    \item Свет от точек сцены проходит через отверстие и проецируется на плоскость проекции, 
    образуя перевернутое изображение сцены.
    \item Наша цель -- построить математическую модель pinhole-камеры, 
    которая лежит в основе большинства методов компьютерного 3D зрения.
\end{itemize}
\end{frame}

%------------------------------------------------------------
\begin{frame}
\frametitle{Pinhole-камера: виртуальное изображение}
\begin{figure}[h]
    \centering
    \includegraphics[width=0.6\textwidth]{inverse_imaging.pdf}
\end{figure}
\begin{itemize}
    \item Неудобно, что изображение перевёрнуто. 
    \item Поэтому мы рассматриваем виртуальное изображение, которое получилось бы, если бы
плоскость изображения располагалась перед отверстием. 
\end{itemize}
\end{frame}

%------------------------------------------------------------
\begin{frame}
\frametitle{Pinhole-камера: терминология}
\begin{figure}[h]
    \centering
    \includegraphics[width=0.6\textwidth]{pinhole_camera.pdf}
\end{figure}
\begin{itemize}
    \item \textbf{optical center} -- точка в которой сходятся все лучи света
    \item \textbf{camera coordinate system} -- система координат, с центром в optical center
    \item \textbf{optical axis} -- ось $w$ системы координат камеры
    \item \textbf{image plane} -- плоскость на которой формируется изображение
    \item \textbf{principal point} -- точка пересечения оптической оси с image plane
    \item \textbf{focal length} -- расстояние от оптического центра до image plane
\end{itemize}
\end{frame}

%------------------------------------------------------------
\begin{frame}
\frametitle{Внутренние параметры камеры}
\begin{figure}[h]
    \centering
    \includegraphics[width=0.5\textwidth]{pinhole_camera.pdf}
    %\includegraphics[width=0.4\textwidth]{focal_params.pdf}
\end{figure}
\begin{itemize}
    \item Из подобия треугольников:
    \begin{align*}
        \frac{x-p_x}{f} = \frac{u}{w}
        \qquad&\Rightarrow\qquad
        x = f\cdot\frac{u}{w} + p_x \\
        \frac{y-p_y}{f} = \frac{v}{w}
        \qquad&\Rightarrow\qquad
        y = f\cdot\frac{v}{w} + p_y 
    \end{align*}
    где $f$ -- фокусное расстояние камеры, $(p_x, p_y)$ -- координата principal point
\end{itemize}
\end{frame}

%------------------------------------------------------------
\begin{frame}
\frametitle{Внутренние параметры камеры}
Формулы проекции трехмерных координат точки из системы координат камеры в систему координат на изображении:
\begin{align*}
        x = f\cdot\frac{u}{w} + p_x \\
        y = f\cdot\frac{v}{w} + p_y 
    \end{align*}
можно записать в однородных координатах в матричной форме:
$$
    \lambda\cdot
    \begin{bmatrix}
        x \\ y \\ 1
    \end{bmatrix} =
    \begin{bmatrix}
        f & 0 & p_x & 0\\
        0 & f & p_y & 0\\
        0 & 0 & 1   & 0
    \end{bmatrix} \cdot
    \begin{bmatrix}
        u \\ v \\ w \\ 1
    \end{bmatrix}
    \,,\qquad \lambda = w
$$
Заметим, что при этом из нелинейных уравнений в декартовых координатах 
мы перешли к линейным уравнениям в однородных координатах.
\end{frame}

%------------------------------------------------------------
\begin{frame}
\frametitle{Внутренние параметры камеры}
\begin{equation} \label{intrinsics}
    \lambda\cdot
    \begin{bmatrix}
        x \\ y \\ 1
    \end{bmatrix} =
    \begin{bmatrix}
        f & 0 & p_x & 0\\
        0 & f & p_y & 0\\
        0 & 0 & 1   & 0
    \end{bmatrix} \cdot
    \begin{bmatrix}
        u \\ v \\ w \\ 1
    \end{bmatrix}
    \qquad\rightarrow\qquad
    \lambda\cdot
    \begin{bmatrix}
        x \\ y \\ 1     
    \end{bmatrix} =
    \begin{bmatrix}
        f_x & s   & c_x & 0\\
        0   & f_y & c_y & 0\\
        0   & 0   & 1   & 0
    \end{bmatrix} \cdot
    \begin{bmatrix}
        u \\ v \\ w \\ 1
    \end{bmatrix}
\end{equation}

\begin{itemize}
    \item Мы предполагали, что оси на изображении перпендикулярны и координаты $x$ и $y$ на изображении
    измеряются в физических единицах (например в миллиметрах).
    \item Обычно координаты на изображении измеряются в пикселях,
    а оси координат могут быть не перпендикулярны.
    \item Если это учесть, то преобразование примет вид \eqref{intrinsics},
    где $f_x$, $f_y$ -- фокусные расстояния в пикселях, 
    $c_x$, $c_y$ -- координаты principal point в пикселях,
    $s$ -- параметр перекоса (skew) осей координат.
    \item Как выражаются $f_x$, $f_y$, $c_x$, $c_y$ через $f$, $p_x$, $p_y$?
\end{itemize}
\end{frame}

%------------------------------------------------------------
\begin{frame}
\frametitle{Внутренние параметры камеры}
\begin{itemize}
    \item \textbf{Матрица внутренних параметров камеры (camera intrinsic matrix)}
    \[
        \mathbf{K} =
        \begin{bmatrix}
            f_x & s   & c_x \\
            0   & f_y & c_y \\
            0   & 0   & 1
        \end{bmatrix}
    \]
    \itemsep0.3cm
    \item Матрица $\mathbf{K}$ содержит пять внутренних параметров камеры:
    \begin{align*}
        &f_x\,, f_y &\quad &\text{-- фокусные расстояния по осям $x$ и $y$ в пикселях} \\
        &c_x\,, c_y &\quad &\text{-- координаты principal point в пикселях} \\
        &s          &\quad &\text{-- параметр перекоса (skew) осей координат}
    \end{align*}
    \item Обычно $s=0$, так как оси координат изображения перпендикулярны и 
    $$
        c_x = \frac{W}{2}\,,\quad
        c_y = \frac{H}{2}
    $$
    где $W$ и $H$ -- ширина и высота изображения в пикселях.
\end{itemize}
\end{frame}

%------------------------------------------------------------
\begin{frame}
\frametitle{Внутренние параметры камеры}
Уравнение проекции координат точки из системы координат камеры в систему координат на изображении
$$
    \lambda\cdot
    \begin{bmatrix}
        x \\ y \\ 1     
    \end{bmatrix} =
    \begin{bmatrix}
        f_x & s   & c_x & 0\\
        0   & f_y & c_y & 0\\
        0   & 0   & 1   & 0
    \end{bmatrix} \cdot
    \begin{bmatrix}
        u \\ v \\ w \\ 1
    \end{bmatrix}
$$ 
можно переписать в матричном виде:
\begin{equation}    \label{camera_matrix_simple}
    \lambda\cdot
    \widetilde{\mathbf{x}} = 
    \left[\mathbf{K}\,\,\mathbf{0}\right] \cdot \widetilde{\mathbf{w}}
\end{equation}
где
\[
    \widetilde{\mathbf{x}} =
    \begin{bmatrix}
        x \\ y \\ 1
    \end{bmatrix}
    \,,\qquad
    \widetilde{\mathbf{w}} =
    \begin{bmatrix}
        u \\ v \\ w \\ 1
    \end{bmatrix}
\]
-- координаты точки в однородных системах координат изображения и камеры соответственно.
\end{frame}

%------------------------------------------------------------
\begin{frame}
\frametitle{Внешние параметры камеры}
\begin{center}
\tdplotsetmaincoords{-60}{-35}
\begin{tikzpicture}[
        tdplot_main_coords,
        scale=0.6,
        >=Stealth,
        scale=1.6,
        description/.style={draw=gray!70, thick, line cap=round, every node/.style={align=center, font=\footnotesize\sffamily, anchor=north}},
    ]   
    \coordinate (o) at (0,0,0);

    \draw [thick, ->, every node/.style={font=\footnotesize, inner sep=1pt}] 
        (o)      edge node [pos=1, anchor=north east] {$w_c$} % 2 -> z
        +(0,1,0) edge node [pos=1, anchor=north east] {$v_c$} % 3 -> y
        +(0,0,1) -- 
        +(1,0,0) node [anchor=north west] {$u_c$} % 1 -> x
    ;

    \path [draw=gray!70, fill=gray!20, opacity=0.8] 
        (-1.5, 4, 1.75) coordinate (a) -- 
        ++(0, 0, -3.5) coordinate (b) -- 
        ++(3, 0, 0) coordinate (c) -- 
        ++(0, 0, 3.5) coordinate (d) -- cycle 
    ;
    
    \draw [thick, green!50!black, <->, shorten >=-15pt, shorten <=-15pt] 
        (a) node [below=15pt, anchor=north, font=\footnotesize] {$y$} -- 
        (b) -- 
        (c) node [above right=17pt, anchor=north west, font=\footnotesize] {$x$}
    ;
    
    \draw [thick, black, fill=black] (b) circle [radius=1.2pt];
    
    \coordinate (q) at (0, 4, 0);
    
    % \draw [thick, cyan, ->, shorten >=-15pt] 
    %     (q) -- ($(d)!1/2!(a)$) 
    %     node [below=15pt, anchor=north] {$v$};
    
    % \draw [thick, cyan, ->, shorten >=-15pt] 
    %     (q) -- ($(d)!1/2!(c)$) 
    %     node [above right=17pt, anchor=north west] {$u$};
    
    \draw [thick, black, fill=black] (q) circle [radius=1.2pt];

    \coordinate (p) at (1,4,-0.5);
    \coordinate (s) at ($1.8*(p)$);

    \draw [thick, fill=red] (p) circle [radius=1.2pt];
    \draw [thick, fill=red] (s) circle [radius=1.2pt];

    \draw [thick, red] (o.center) -- (p.center);
    \scoped[on background layer]{
        \draw [red] (p.center) -- (s.center);
    }

    \draw [thick, gray] (o.center) -- (q.center);
    \scoped[on background layer]{
        \draw [thick, gray] (q.center) -- (0, 5, 0);
    }

    \draw [thick, red] (o.center) -- (p.center);
    \draw [thick, black, fill=black] (o) circle [radius=1.2pt];

    \path [every node/.style={font=\footnotesize, inner sep=2pt}] 
        (o) node [below right] {$\mathbf{c}$};
    
    \path [every node/.style={font=\scriptsize, inner sep=2pt}] 
        (s) node [above right] {$\mathbf{w}=[u,v,w]$};

    \path [every node/.style={font=\scriptsize, inner sep=2pt}] 
        (p) node [above right] {$\mathbf{x}$};

    \path [description] (0,0,0) [out=70, in=95] to (2, 0, -2) node {camera\\coordinate\\system};

    \tdplotsetrotatedcoords{10}{40}{30}
    \coordinate (Shift) at (5,-2,-3);
    \tdplotsetrotatedcoordsorigin{(Shift)}
    
    \draw [thick, black, fill=black] (Shift) circle [radius=1.2pt];
    
    \draw [tdplot_rotated_coords, thick, ->, every node/.style={font=\footnotesize, inner sep=1pt}] 
        (0,0,0) edge node [pos=1, anchor=north east] {$w$} % 2 -> z
        +(0,1,0) edge node [pos=1, anchor=north east] {$v$} % 3 -> y
        +(0,0,1) -- 
        +(1,0,0) node [anchor=north west] {$u$} % 1 -> x
    ;

    \path [description] (Shift) [out=-95, in=95] to (4, -2, -2) node {world\\coordinate\\system};
\end{tikzpicture}
\end{center}
\begin{itemize}
    \item Положение точки в пространстве обычно задается относительно 
    мировой системы координат, которая не совпадает с системой координат камеры.
\end{itemize}
\end{frame}

%------------------------------------------------------------
\begin{frame}
\frametitle{Внешние параметры камеры}
\begin{itemize}
    \item Преобразование координат из мировой системы координат
в систему координат камеры задаётся с помощью
матрицы вращения $\mathbf{R}$ и вектора сдвига $\mathbf{t}$:
\begin{equation} \label{R_t_homogeneous}
    \widetilde{\mathbf{w}}_c =
    \begin{bmatrix}
        \mathbf{R} & \mathbf{t} \\
        \mathbf{0} & 1
    \end{bmatrix}
    \cdot
    \widetilde{\mathbf{w}}
\end{equation}
где
\[
    \widetilde{\mathbf{w}} =
    \begin{bmatrix}
        u & v & w & 1
    \end{bmatrix}^\top\,,
    \qquad
    \widetilde{\mathbf{w}}_c =
    \begin{bmatrix}
        u_c & v_c & w_c & 1
    \end{bmatrix}^\top
\]
-- однородные координаты точки в мировой системе координат и системе координат камеры.
\itemsep0.3cm
\item Матрица вращения $\mathbf{R}$ и вектор сдвига $\mathbf{t}$  
называются \textbf{внешними параметрами камеры (camera extrinsic parameters)}.
\itemsep0.3cm
\item Всего внешних параметров камеры шесть: три параметра матрицы вращения $\mathbf{R}$
и три параметра вектора сдвига $\mathbf{t}$.
\end{itemize}
\end{frame}

%------------------------------------------------------------
\begin{frame}
\frametitle{Модель pinhole-камеры}
Учитывая уравнение \eqref{R_t_homogeneous} перехода из мировой системы координат в систему координат камеры
\[
    \widetilde{\mathbf{w}}_c =
    \begin{bmatrix}
        \mathbf{R} & \mathbf{t} \\
        \mathbf{0} & 1
    \end{bmatrix}
    \cdot
    \widetilde{\mathbf{w}}
\]
 и уравнение \eqref{camera_matrix_simple} перехода из системы координат камеры в систему координат изображения
\[
    \lambda\cdot
    \widetilde{\mathbf{x}} = \left[\mathbf{K}\,\,\mathbf{0}\right] \cdot \widetilde{\mathbf{w}}_c
\]
можно записать уравнение, описывающее полный переход из мировой системы координат в систему координат изображения:
\[
    \lambda\cdot
    \widetilde{\mathbf{x}} =
    \left[\mathbf{K}\,\,\mathbf{0}\right] \cdot
    \begin{bmatrix}
        \mathbf{R} & \mathbf{t} \\
        \mathbf{0} & 1
    \end{bmatrix} \cdot
    \widetilde{\mathbf{w}}      
\]
или после умножения матриц: 
\begin{equation}  \label{projective_camera_model}   
    \lambda\cdot
    \widetilde{\mathbf{x}} =
    \mathbf{K} \cdot
    \begin{bmatrix}
        \mathbf{R} & \mathbf{t}
    \end{bmatrix} \cdot
    \widetilde{\mathbf{w}}
\end{equation}
Уравнение \eqref{projective_camera_model} задает \textbf{модель pinhole-камеры}.
\end{frame}

%------------------------------------------------------------
\begin{frame}
\frametitle{Матрица камеры}
\begin{itemize}
    \item В модели pinhole-камеры
    $$
        \lambda\cdot
        \widetilde{\mathbf{x}} =
        \mathbf{K} \cdot
        \begin{bmatrix}
            \mathbf{R} & \mathbf{t}
        \end{bmatrix} \cdot
        \widetilde{\mathbf{w}}
    $$
    \item $3\times4$ матрица
    \[
        \mathbf{P} =
        \mathbf{K} \cdot
        \begin{bmatrix}
            \mathbf{R} & \mathbf{t}
        \end{bmatrix}
    \]  
    называется \textbf{матрицей камеры (camera matrix)} 
    и содержит все внутренние и внешние параметры камеры.
    \itemsep0.3cm
    \item Всего матрица $\mathbf{P}$ содержит 11 степеней свободы:
    \begin{itemize}
        \item пять внутренних параметров камеры из матрицы $\mathbf{K}$
        \item шесть внешних параметров камеры из матрицы $\left[\mathbf{R}\,\,\mathbf{t}\right]$
    \end{itemize}
\end{itemize}
\end{frame}

%------------------------------------------------------------
\begin{frame}
\frametitle{Проектирование в модели pinhole-камеры}
\begin{center}
\tdplotsetmaincoords{-60}{-35}
\begin{tikzpicture}[
        tdplot_main_coords,
        scale=0.5,
        >=Stealth,
        scale=1,
        description/.style={draw=gray!70, thick, line cap=round, every node/.style={align=center, font=\footnotesize\sffamily, anchor=north}},
    ]   
    \coordinate (o) at (0,0,0);

    \draw [thick, ->, every node/.style={font=\footnotesize, inner sep=1pt}] 
        (o)      edge node [pos=1, anchor=north east] {$w_c$} % 2 -> z
        +(0,1,0) edge node [pos=1, anchor=north east] {$v_c$} % 3 -> y
        +(0,0,1) -- 
        +(1,0,0) node [anchor=north west] {$u_c$} % 1 -> x
    ;

    \path [draw=gray!70, fill=gray!20, opacity=0.8] 
        (-1.5, 4, 1.75) coordinate (a) -- 
        ++(0, 0, -3.5) coordinate (b) -- 
        ++(3, 0, 0) coordinate (c) -- 
        ++(0, 0, 3.5) coordinate (d) -- cycle 
    ;
    
    \draw [thick, green!50!black, <->, shorten >=-15pt, shorten <=-15pt] 
        (a) node [below=15pt, anchor=north, font=\footnotesize] {$y$} -- 
        (b) -- 
        (c) node [above right=17pt, anchor=north west, font=\footnotesize] {$x$}
    ;
    
    \draw [thick, black, fill=black] (b) circle [radius=1.2pt];
    
    \coordinate (q) at (0, 4, 0);
    
    \draw [thick, black, fill=black] (q) circle [radius=1.2pt];

    \coordinate (p) at (1,4,-0.5);
    \coordinate (s) at ($1.8*(p)$);

    \draw [thick, fill=red] (p) circle [radius=1.2pt];
    \draw [thick, fill=red] (s) circle [radius=1.2pt];

    \draw [thick, red] (o.center) -- (p.center);
    \scoped[on background layer]{
        \draw [red] (p.center) -- (s.center);
    }

    \draw [thick, gray] (o.center) -- (q.center);
    \scoped[on background layer]{
        \draw [thick, gray] (q.center) -- (0, 5, 0);
    }

    \draw [thick, red] (o.center) -- (p.center);
    \draw [thick, black, fill=black] (o) circle [radius=1.2pt];

    \path [every node/.style={font=\footnotesize, inner sep=2pt}] 
        (o) node [below right] {$\mathbf{c}$};
    
    \path [every node/.style={font=\scriptsize, inner sep=2pt}] 
        (s) node [above right] {$\mathbf{w}=[u,v,w]$};

    \path [every node/.style={font=\scriptsize, inner sep=2pt}] 
        (p) node [above right] {$\mathbf{x}$};

    \tdplotsetrotatedcoords{10}{40}{30}
    \coordinate (Shift) at (5,-2,-3);
    \tdplotsetrotatedcoordsorigin{(Shift)}
    
    \draw [thick, black, fill=black] (Shift) circle [radius=1.2pt];
    
    \draw [tdplot_rotated_coords, thick, ->, every node/.style={font=\footnotesize, inner sep=1pt}] 
        (0,0,0) edge node [pos=1, anchor=north east] {$w$} % 2 -> z
        +(0,1,0) edge node [pos=1, anchor=north east] {$v$} % 3 -> y
        +(0,0,1) -- 
        +(1,0,0) node [anchor=north west] {$u$} % 1 -> x
    ;
\end{tikzpicture}
\end{center}

Проекция точки $\widetilde{\mathbf{w}}$ из мировой системы координат в систему координат изображения:
    $$
        \lambda\cdot
        \widetilde{\mathbf{x}} =
        \mathbf{P} \cdot
        \widetilde{\mathbf{w}}
    $$
Можно показать, что обратно, точка на изображении проектируется в луч:
    $$
        \widetilde{\mathbf{w}} = \mathbf{P}^+ \cdot
        \widetilde{\mathbf{x}}  + t \cdot \widetilde{\mathbf{c}}
        \qquad
        t\in\mathbb{R}
    $$
где $\mathbf{P}^+=\mathbf{P}^\top(\mathbf{P}\mathbf{P}^\top)^{-1}$ 
-- псевдообратная матрица камеры,
$\widetilde{\mathbf{c}}$ -- однородные координаты оптического центра камеры,
$t$ -- произвольный параметр.

\end{frame}

%------------------------------------------------------------
\begin{frame}
\frametitle{Дисторсия камеры}
\begin{figure}[h]
    \centering
    \includegraphics[width=0.3\textwidth]{distortion.jpg}
    \hspace{0.3cm}
    \includegraphics[width=0.21\textwidth]{fisheye_camera.jpg}
    \includegraphics[width=0.21\textwidth]{fisheye_image.jpg}
\end{figure}
\begin{itemize}
    \item Реальные камеры отклоняются от идеальной модели pinhole-камеры
    из-за оптических искажений (дисторсии) объектива камеры.
    \item Наиболее распространены радиальная и тангенциальная дисторсии.
    \item Дисторсия особенно заметна на широкоугольных и fisheye-объективах.
    \item Для того чтобы пользоваться pinhole-моделью камеры необходима
    коррекция дисторсии на изображении.
\end{itemize}
\end{frame}

%------------------------------------------------------------
\begin{frame}
\frametitle{Модель радиальной дисторсии камеры}
\begin{figure}[h]
    \centering
    \includegraphics[width=0.7\textwidth]{radial_distortion.png}
\end{figure}
\begin{itemize}
    \item Радиальная дисторсия возникает из-за несовершенств объектива камеры.
    \item Радиальная дисторсия зависит от расстояния пикселя от principal point на изображении
    $$
        r = \sqrt{(x - c_x)^2 + (y - c_y)^2}
    $$
    \item Модель радиальной дисторсии с коэффициентами $k_1$, $k_2$, $k_3$:
    \begin{align*}
        x_{d} &= x \cdot (1 + k_1 r^2 + k_2 r^4 + k_3 r^6) \\
        y_{d} &= y \cdot (1 + k_1 r^2 + k_2 r^4 + k_3 r^6)
    \end{align*}
\end{itemize}
\end{frame}

%------------------------------------------------------------
\begin{frame}
\frametitle{Модель тангенциальной дисторсии камеры}
\begin{figure}[h]
    \centering
    \includegraphics[width=0.4\textwidth]{tangent_distortion.png}
\end{figure}
\begin{itemize}
    \item Тангенциальная дисторсия возникает из-за наклона линз объектива камеры к плоскости изображения.
    \item Модель тангенциальной дисторсии с коэффициентами $p_1$, $p_2$:
    \begin{align*}
        x_{d} &= x + [2 p_1 x y + p_2 (r^2 + 2 x^2)] \\
        y_{d} &= y + [p_1 (r^2 + 2 y^2) + 2 p_2 x y]
    \end{align*}
\end{itemize}
\end{frame}

%------------------------------------------------------------
\begin{frame}
\frametitle{Калибровка радиальной и тангенциальной дисторсии}
\begin{figure}[h]
    \centering
    \includegraphics[width=0.6\textwidth]{undistortion.png}
\end{figure}
\begin{itemize}
        \item Коэффициенты радиальной и тангенциальной дисторсий $(k_1,k_2,k_3,p_1,p_2)$ 
        а также внутренние параметры камеры $\mathbf{K}$
        можно найти с помощью процедуры \textbf{калибровки камеры}.
        \itemsep0.3cm
        \item Зная коэффициенты дисторсии, можно найти \textbf{undistort rectify map} $(m_x, m_y)$
        и скорректировать дисторсию:
        $$
            x = x_d + m_x\,,\qquad
            y = y_d + m_y
        $$

        \item После коррекции дисторсии, можно использовать pinhole-модель камеры.
\end{itemize}
\end{frame}

%------------------------------------------------------------
\begin{frame}
\frametitle{Теоретические вопросы}
\begin{itemize}
    \item Однородные координаты в 2D и 3D.
    \vspace{0.3cm}
    
    \item Замена системы координат с помощью матрицы вращения и вектора сдвига
 (формулы замены в декартовых и однородных координатах).
    \vspace{0.3cm}

    \item Параметризация матрицы вращения с помощью вектора вращения. Формулы Родрига.
    \vspace{0.3cm}
    
    \item Модель pinhole-камеры. Внутренние и внешние параметры камеры.
    \vspace{0.3cm}
    
    \item Матрица камеры. Проекция 3D точек на изображение с помощью матрицы камеры и обратная проекция точек изображения в 3D пространство.
    \vspace{0.3cm}
    
    \item Модель радиальной и тангенциальной дисторсии камеры.
\end{itemize}
\end{frame}

%------------------------------------------------------------
\end{document}

