\documentclass[12pt, usepdftitle=false, aspectratio=1610]{beamer}

\usetheme{Madrid}

\usefonttheme[]{serif}
\setbeamertemplate{caption}[numbered]
\setbeamertemplate{navigation symbols}{}

\usepackage[T2A]{fontenc}			
\usepackage[utf8]{inputenc}			
\usepackage[english,russian]{babel}

\usepackage{
    mathtext,
    minted,
    cmap,
    multirow,
    textcomp,
    graphicx,
    wrapfig,
    subfig,
    mathtools,
    gensymb,
    amsmath,
    hyperref
}
\usepackage[font=small,labelfont=bf]{caption}
\usepackage[absolute, overlay]{textpos}

\DeclarePairedDelimiter{\norm}{\lVert}{\rVert} 
\DeclareMathOperator*{\argmax}{arg\,max}
\DeclareMathOperator*{\argmin}{arg\,min}

\graphicspath{{./figs/06}}

\title[Лекция 6]{Сегментация и морфологическая обработка изображений}

\author{Александр Танченко}
\institute{}
\date{2025}

\begin{document}

%------------------------------------------------------------
\begin{frame}
    \titlepage
\end{frame}

%------------------------------------------------------------
\begin{frame}
\frametitle{Задачи сегментации и морфологической обработки}
\begin{itemize}
    \item \textbf{Сегментация изображений} -- присвоить пикселям метки классов
    \begin{figure}
        \centering
        \includegraphics[height=0.3\textheight]{satellite.png}
        \hspace{0.3cm}
        \includegraphics[height=0.3\textheight]{semantic_road.png}
    \end{figure}
    \vspace*{0.5cm}
    \item \textbf{Морфологическая обработка изображений} -- операции над бинарными изображениями
    \begin{figure}
        \centering
        \includegraphics[height=0.3\textheight]{seg_morph.png}
    \end{figure}
\end{itemize}
\end{frame}

%------------------------------------------------------------
\begin{frame}
\frametitle{Примеры использования}
\begin{figure}
    \centering
    \includegraphics[height=0.35\textheight]{road_seg.jpg}
    \includegraphics[height=0.35\textheight]{sat_segm.png}
\end{figure}
\begin{figure}
    \centering
    \includegraphics[height=0.35\textheight]{med_seg.png}
    \includegraphics[height=0.35\textheight]{med_seg1.png}
\end{figure}
\end{frame}

%------------------------------------------------------------
\begin{frame}
\frametitle{Сегментация с помощью обучения с учителем}
\begin{figure}
    \centering
    \includegraphics[width=0.9\textwidth]{segm_types.pdf}
\end{figure}
\begin{itemize}
    \item[(b)] \textbf{semantic segmentation} -- каждому пикселю присваивается метка класса
    \item[(c)] \textbf{instance segmentation} -- не только классифицировать пиксели по классу, но и отличать друг от друга отдельные экземпляры объектов одного класса
    \item[(d)] \textbf{panoptic segmentation} -- объединяет semantic и instance сегментацию
\end{itemize}
\end{frame}

%------------------------------------------------------------
\begin{frame}
\frametitle{Сегментация с помощью кластеризации}
Пиксели изображения кластеризуются по признакам (цвет, яркость, координаты, текстура и т.д.)

\begin{itemize}
    \item \textbf{пороговая сегментация}
    \item \textbf{k-means сегментация}
    \item \textbf{mean-shift сегментация}
\end{itemize}
\end{frame}

%---------------------------------------------------
\begin{frame}
\frametitle{Пороговая сегментация}
\begin{figure}
    \centering
    \includegraphics[width=0.3\textwidth]{seg_threshold1.pdf}
    \includegraphics[width=0.3\textwidth]{seg_threshold2.pdf}
    \includegraphics[width=0.3\textwidth]{seg_threshold3.pdf}
\end{figure}

Сегментация с порогами $T_1$ и $T_2$
$$
    g(x, y) =
    \begin{cases}
        0\,, & f(x,y) < T_1 \\        
        128\,, & T_1 \leqslant f(x,y) \leqslant T_2 \\
        255\,, & f(x,y) > T_2 
    \end{cases}
$$
\end{frame}

%---------------------------------------------------
\begin{frame}
\frametitle{Метод Оцу вычисления порога бинарной сегментации}
\begin{figure}
    \centering
    \includegraphics[width=0.4\textwidth]{otsu.pdf}
\end{figure}

\begin{itemize}
    \item для каждого порога $T=1,2,\ldots,255$, разобьем все пиксели на два класса
    $$
        C_1 = \left\{(x,y): f(x,y) < T\right\}\,,\quad
        C_2 = \left\{(x,y): f(x,y) \geqslant T\right\}
    $$
    \item в методе Оцу порог подбирается так, что бы дисперсия значений пикселей в каждом классе была минимальна
    $$
        T_\ast=\argmin_{0 < T < 256} 
        \left[\sigma_1^2  + \sigma_2^2\right]
    $$
    где $\sigma_1^2, \sigma_2^2$ -- дисперсии яркостей пикселей в классах $C_1$ и $C_2$
\end{itemize}
\end{frame}

%---------------------------------------------------
\end{document}



%------------------------------------------------------------
\begin{frame}
\frametitle{}
\end{frame}