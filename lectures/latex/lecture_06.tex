\documentclass[12pt, usepdftitle=false, aspectratio=1610]{beamer}

\usetheme{Madrid}

\usefonttheme[]{serif}
%\usefonttheme{structuresmallcapsserif}

\setbeamertemplate{caption}[numbered]
\setbeamertemplate{navigation symbols}{}

\usepackage[T2A]{fontenc}			
\usepackage[utf8]{inputenc}			
\usepackage[english,russian]{babel}

\usepackage{
    mathtext,
    minted,
    cmap,
    multirow,
    textcomp,
    graphicx,
    wrapfig,
    subfig,
    mathtools,
    gensymb,
    amsmath,
    hyperref,
    subfig
}
\usepackage[absolute, overlay]{textpos}
\usepackage[font=small,labelfont=bf]{caption}
\usepackage{subcaption}

\DeclarePairedDelimiter{\norm}{\lVert}{\rVert} 
\DeclareMathOperator*{\argmax}{arg\,max}
\DeclareMathOperator*{\argmin}{arg\,min}

\graphicspath{{./figs/06}}

\title[Лекция 6]{Сегментация изображений}

\author{Александр Танченко}
\institute{}
\date{2025}

\begin{document}

%------------------------------------------------------------
\begin{frame}
    \titlepage
\end{frame}

%------------------------------------------------------------
\begin{frame}
    \frametitle{Задача сегментации изображений}
    \begin{figure}
        \captionsetup[subfigure]{labelformat=empty}
        \centering
        \subfloat{\includegraphics[height=0.3\textheight]{satellite.png}}
        \qquad
        \subfloat{\includegraphics[height=0.3\textheight]{semantic_road.png}}
    \end{figure}
    \begin{figure}
        \captionsetup[subfigure]{labelformat=empty}
        \centering
        \subfloat{\includegraphics[height=0.3\textheight]{meanshift13.png}}
        \qquad
        \subfloat{\includegraphics[height=0.3\textheight]{meanshift14.png}}
    \end{figure}
    \textbf{Сегментация изображений} -- объединение пикселей по каким-либо признакам (цвет, координаты, метки классов и т.д.)
\end{frame}

%------------------------------------------------------------
\begin{frame}
\frametitle{Примеры использования}
\begin{figure}
    \centering
    \includegraphics[height=0.35\textheight]{road_seg.jpg}
    \includegraphics[height=0.35\textheight]{sat_segm.png}
\end{figure}
\begin{figure}
    \centering
    \includegraphics[height=0.35\textheight]{med_seg.png}
    \includegraphics[height=0.35\textheight]{med_seg1.png}
\end{figure}
\end{frame}

%------------------------------------------------------------
\begin{frame}
\frametitle{Сегментация с помощью классификации}
\begin{figure}
    \centering
    \includegraphics[width=0.9\textwidth]{segm_types.pdf}
\end{figure}
Каждому пикселю изображения присваивается метка класса.
\vspace{0.2cm}

Виды сегментации:
\vspace{0.2cm}
\begin{itemize}
    \item[(b)] \textbf{semantic segmentation} -- каждому пикселю присваивается метка класса
    \item[(c)] \textbf{instance segmentation} -- не только классифицировать пиксели по классу, но и отличать друг от друга отдельные экземпляры объектов одного класса
    \item[(d)] \textbf{panoptic segmentation} -- объединяет semantic и instance сегментацию
\end{itemize}
\end{frame}

%------------------------------------------------------------
\begin{frame}
\frametitle{Сегментация с помощью кластеризации}
\begin{figure}
    \captionsetup[subfigure]{labelformat=empty}
    \centering
    \subfloat{\includegraphics[height=0.3\textheight]{meanshift13.png}}
    \qquad
    \subfloat{\includegraphics[height=0.3\textheight]{meanshift15.png}}
    \caption*{Пример сегментации с помощью признаков $\boldsymbol{f}=\left(R\,,G\,,B\,,x\,,y\right)$}
\end{figure}

Пиксели изображения кластеризуются по признакам.
\vspace{0.2cm}

В качестве признаков можно использовать:
\begin{itemize}
    \item цветовые каналы (RGB, HSV, YCrCb и т.д.)
    \vspace{0.2cm}
    \item координаты пикселей $(x, y)$
    \vspace{0.2cm}
    \item признаки, выделенные нейронными сетями
\end{itemize}
\end{frame}

%------------------------------------------------------------
\begin{frame}
\frametitle{Сегментация с помощью кластеризации}

Способы сегментации
\vspace{0.4cm}

\begin{itemize}
    \item Сегментация с помощью фундаментальных моделей
    \begin{itemize}
        \item SAM2  (Segment Anything 2, 11M images, 1B+ masks) \url{https://sam2.metademolab.com/demo}
        \vspace{0.2cm}
        \item DINO2 \url{https://dinov2.metademolab.com/demos}
    \end{itemize}
    \vspace{0.2cm}

    \item Классические методы сегментации
    \begin{itemize}
        \item \textbf{пороговая сегментация}
        \vspace{0.2cm}
        \item \textbf{k-means сегментация}
        \vspace{0.2cm}
        \item \textbf{mean-shift сегментация}
    \end{itemize}
\end{itemize}
\end{frame}

%---------------------------------------------------
\begin{frame}
\frametitle{Пороговая сегментация}
\begin{figure}
    \centering
    \includegraphics[width=0.3\textwidth]{seg_threshold1.pdf}
    \includegraphics[width=0.3\textwidth]{seg_threshold2.pdf}
    \includegraphics[width=0.3\textwidth]{seg_threshold3.pdf}
\end{figure}

Сегментация с порогами $T_1$ и $T_2$
$$
    g(x, y) =
    \begin{cases}
        0\,, & f(x,y) < T_1 \\        
        128\,, & T_1 \leqslant f(x,y) \leqslant T_2 \\
        255\,, & f(x,y) > T_2 
    \end{cases}
$$
\end{frame}

%---------------------------------------------------
\begin{frame}
\frametitle{Метод Оцу вычисления порога бинарной сегментации}
\begin{figure}
    \centering
    \includegraphics[width=0.4\textwidth]{otsu.pdf}
\end{figure}

\begin{itemize}
    \item для каждого порога $T=1,2,\ldots,255$, разобьем все пиксели на два класса
    $$
        C_1 = \left\{(x,y): f(x,y) < T\right\}\,,\quad
        C_2 = \left\{(x,y): f(x,y) \geqslant T\right\}
    $$
    \item в методе Оцу порог подбирается так, что бы дисперсия значений пикселей в каждом классе была минимальна
    $$
        T_\ast=\argmin_{0 < T < 256} 
        \left[\sigma_1^2(T)  + \sigma_2^2(T)\right]
    $$
    где $\sigma_1^2, \sigma_2^2$ -- дисперсии яркостей пикселей в классах $C_1$ и $C_2$
\end{itemize}
\end{frame}

%---------------------------------------------------------------------
\begin{frame}
\frametitle{Пример: 3-means кластеризация}
\begin{wrapfigure}{r}{0.3\textwidth}
    \vspace{-40pt}
    \begin{center}
        \includegraphics[height=0.85\textheight]{3means.png}
    \end{center}
\end{wrapfigure}

\textbf{Задача:} разбить точки в \textbf{пространстве признаков} на $3$ кластера
\vspace*{1cm}

\textbf{Решение:}
\vspace*{0.5cm}
\begin{itemize}
    \item[\textbf{1.}] случайно инициализировать центры кластеров 
    \vspace*{0.2cm}
    \item[\textbf{2.}] создать кластеры, отнеся каждую точку к ближайшему центру
    \vspace*{0.2cm}
    \item[\textbf{3.}] пересчитать центры кластеров как среднее значение точек в кластере
    \vspace*{0.2cm}
    \item[\textbf{4.}] повторять шаги 2-3 пока центры кластеров не перестанут меняться
\end{itemize}
\end{frame}

%---------------------------------------------------------------------
\begin{frame}
\frametitle{k-means кластеризация}

\textbf{Дано:} $N$ точек в пространстве признаков $\boldsymbol{f}_n\in\mathbf{R}^d$ и число кластеров $k$
\vspace*{0.5cm}

\textbf{Задача:} найти $k$ кластеров
\vspace*{0.5cm}

\textbf{Решение:}
\vspace*{0.2cm}
\begin{itemize}
    \item[\textbf{1.}] случайно инициализировать $k$ центров кластеров $\boldsymbol{c}_1,\ldots,\boldsymbol{c}_k\in\mathbf{R}^d$
    \vspace*{0.2cm}
    \item[\textbf{2.}] для каждой точки $\boldsymbol{f}_n$ найти ближайший центр кластера $\boldsymbol{c}_i$ и отнести точку к  кластеру $i$
    \vspace*{0.2cm}
    \item[\textbf{3.}] пересчитать центры кластеров как среднее значение точек в кластере
    \vspace*{0.2cm}
    \item[\textbf{4.}] повторять шаги 2-3 пока изменения во всех центрах кластеров не станут меньше $\varepsilon$
\end{itemize}
\end{frame}

%------------------------------------------------------------
\begin{frame}
\frametitle{Свойства k-means кластеризации}
\begin{itemize}
    \item простой и быстрый алгоритм
    \vspace*{0.2cm}
    \item требует задания числа кластеров $k$
    \vspace*{0.2cm}
    \item  чувствителен к выбору начальных центров кластеров
    \vspace*{0.2cm}
    \item чувствителен к выбросам
\end{itemize}
\end{frame}

%------------------------------------------------------------
\begin{frame}
    \frametitle{k-means сегментация изображения}
    \begin{figure}
        \captionsetup[subfigure]{labelformat=empty}
        \centering
        \subfloat[Исходное изображение]{\includegraphics[height=0.3\textheight]{kmeans5.jpg}}
        \qquad
        \subfloat[$\boldsymbol{f}=\left(R\,,G\,,B\right)$]{\includegraphics[height=0.3\textheight]{kmeans8.jpg}}
    \end{figure}
    \begin{figure}
        \captionsetup[subfigure]{labelformat=empty}
        \centering
        \subfloat[$k=16$, $\boldsymbol{f}=\left(R\,,G\,,B\right)$]{\includegraphics[height=0.3\textheight]{kmeans6.jpg}}
        \qquad
        \subfloat[$\boldsymbol{f}=\left(R\,,G\,,B\,,x\,,y\right)$]{\includegraphics[height=0.3\textheight]{kmeans7.jpg}}
    \end{figure}
\end{frame}


%---------------------------------------------------------------------
\begin{frame}
    \frametitle{Mean-shift кластеризация}
    \begin{figure}
        \captionsetup[subfigure]{labelformat=empty}
        \centering
        \subfloat[Точки признаков]{\includegraphics[height=0.35\textheight]{meanshift1.pdf}}
        \quad
        \subfloat[Плотность распределения признаков]{\includegraphics[height=0.35\textheight]{meanshift3.pdf}}
        \quad
        \subfloat[Кластеры]{\includegraphics[height=0.35\textheight]{meanshift2.pdf}}
    \end{figure}
    \begin{itemize}
        \item каждый холм представляет кластер
        \item пики холмов -- центры кластеров
        \item каждая точка взбирается по самому крутому холму 
        \item точки, взобравшиеся на один холм, образуют кластер
    \end{itemize}
\end{frame}

%------------------------------------------------------------
\begin{frame}
\frametitle{Как точки взбираются на холмы}
\begin{figure}
    \centering
    \subfloat{\includegraphics[height=0.3\textheight]{meanshift4.png}}
    \quad
    \subfloat{\includegraphics[height=0.3\textheight]{meanshift5.png}}
    \quad
    \subfloat{\includegraphics[height=0.3\textheight]{meanshift6.png}}
\end{figure}
\begin{itemize}
    \item каждая точка $\boldsymbol{f}_n$ сдвигается в сторону среднего значения точек в окрестности радиуса $R$
\end{itemize}
\end{frame}

%------------------------------------------------------------
\begin{frame}
\frametitle{Как точки взбираются на холмы}
\begin{figure}
    \centering
    \subfloat{\includegraphics[height=0.3\textheight]{meanshift7.png}}
    \quad
    \subfloat{\includegraphics[height=0.3\textheight]{meanshift8.png}}
    \quad
    \subfloat{\includegraphics[height=0.3\textheight]{meanshift9.png}}
\end{figure}
\begin{itemize}
    \item каждая точка $\boldsymbol{f}_n$ сдвигается пока не перестанет меняться
    \item точки, которые сойдутся в одну точку, образуют кластер
\end{itemize}
\end{frame}

%------------------------------------------------------------
\begin{frame}
\frametitle{Свойства mean-shift кластеризации}
\begin{itemize}
    \item простой алгоритм, но вычислительно дорогой 
    \vspace*{0.2cm}
    \item находит число кластеров автоматически
    \vspace*{0.2cm}
    \item  не требует инициализации центров кластеров
    \vspace*{0.2cm}
    \item нечувствителен к выбросам
    \vspace*{0.2cm}
    \item кластеризация зависит от радиуса окрестности $R$
\end{itemize}
\end{frame}

%------------------------------------------------------------
\begin{frame}
    \frametitle{Mean-shift vs k-means: кластеризация точек}
    \begin{figure}
        \captionsetup[subfigure]{labelformat=empty}
        \centering
        \subfloat[Точки на плоскости]{\includegraphics[height=0.45\textheight]{meanshift10.png}}
        \qquad
        \subfloat[k-means, $k=3$]{\includegraphics[height=0.45\textheight]{meanshift11.png}}
        \qquad
        \subfloat[Mean shift]{\includegraphics[height=0.45\textheight]{meanshift12.png}}
    \end{figure}
\end{frame}

%------------------------------------------------------------
\begin{frame}
    \frametitle{Mean-shift vs k-means: сегментация изображения}
    \begin{figure}
        \captionsetup[subfigure]{labelformat=empty}
        \centering
        \subfloat[$\boldsymbol{f}=\left(R\,,G\,,B\,,x\,,y\right)$]{\includegraphics[height=0.3\textheight]{meanshift13.png}}
        \qquad
        \subfloat[k-means, $k=16$]{\includegraphics[height=0.3\textheight]{meanshift14.png}}
        \qquad
        \subfloat[Mean shift]{\includegraphics[height=0.3\textheight]{meanshift15.png}}
    \end{figure}
\end{frame}

%---------------------------------------------------
\end{document}


%------------------------------------------------------------
\begin{frame}
\frametitle{}
\end{frame}