\documentclass[
    12pt, 
    usepdftitle=false,
    aspectratio=1610
]{beamer}

\usetheme{Madrid}

\usefonttheme[]{serif}
\setbeamertemplate{caption}[numbered]
\setbeamertemplate{navigation symbols}{}

\usepackage[T2A]{fontenc}			
\usepackage[utf8]{inputenc}			
\usepackage[english,russian]{babel}

\usepackage{
    mathtext,
    minted,
    cmap,
    multirow,
    textcomp,
    graphicx,
    wrapfig,
    subfig,
    mathtools,
    gensymb
}
\usepackage[font=small,labelfont=bf]{caption}

\title[Лекция 1]{
    Цветовые и геометрические операции над изображениями
}

\author{Александр Танченко}
\institute{}
\date{2025}

\begin{document}

%------------------------------------------------------------
\begin{frame}

\textbf{Преподаватели}
\begin{itemize}
    \item лекции: Александр Танченко
    \item практика: Александр Кодуков
\end{itemize}
\vspace{1cm}

\textbf{Прислать на почту alex.tanch@gmail.com}
\begin{itemize}
    \item ФИО
    \item telegram username
    \item github username
\end{itemize}
\end{frame}

%------------------------------------------------------------
\begin{frame}    
    \textbf{Курсы по CV}
    \vspace{0.2cm}
    \begin{itemize}
    \item Анализ изображений
    \item Глубокое обучение в задачах компьютерного зрения
    \item Обработка видео
    \end{itemize}
\end{frame}

%------------------------------------------------------------
\begin{frame}

\textbf{Содержание курса <<Анализ изображений>>}
\vspace{0.2cm}

\begin{itemize}
    \item Цветовые и геометрические операции над изображениями
    \item Пространственная и частотная обработка изображений
    \item Методы улучшения качества изображений
    \item Детекция границ на изображениях
    \item Сегментация и морфологическая обработка изображений
    \item Детекция и сопоставление ключевых точек на изображениях
    \item Гомография. Сшивка изображений
    \item Проективная модель камеры. Калибровка камеры
    \item Эпиполярная геометрия. Относительное положение камер
    \item Реконструкция 3D сцены
    \item Методы понижения размерности многомерных данных. Классификация изображений
\end{itemize}
\end{frame}

%------------------------------------------------------------
\begin{frame}

\textbf{Python}
\begin{itemize}
    \item numpy, scipy
    \item opencv
    \item matplotlib, plotly
\end{itemize}
\vspace{0.5cm}

\textbf{Математика}
\begin{itemize}
    \item 2D и 3D геометрия, матрицы, системы, SVD и QR разложения
    \item случайные величины, метод максимального правдоподобия
    \item методы Гаусса-Ньютона, Левенберга-Марквардта
\end{itemize}
\vspace{0.5cm}

\textbf{Книги}
\begin{itemize}
    \item Гонсалес, Вудс <<Цифровая обработка изображений>>
    \item Hartley, Zisserman <<Multiple View Geometry in Computer Vision>>
    \item Кэлер, Брэдски <<Изучаем OpenCV3>>
\end{itemize}
\end{frame}

%------------------------------------------------------------
\begin{frame}
    \url{https://github.com/alextanch/image-processing}
    \vspace{0.5cm}

    \hspace{1cm}
    \url{./lectues} \hspace{1.3cm} лекции
    \vspace{0.2cm}

    \hspace{1cm}
    \url{./notebooks} \hspace{0.85cm} практика
    \vspace{0.2cm}
    
    % \hspace{1cm}
    % \url{./tasks} \hspace{1.8cm} задачи
    % \vspace{0.2cm}
    
    \hspace{1cm}
    \url{./data} \hspace{2cm} данные
    \vspace{0.2cm}

    \hspace{1cm}
    \url{./books} \hspace{1.8cm} книги
    \vspace{0.2cm}

    \hspace{1cm}
    \url{./README.md}   \hspace{0.9cm} установка python окружения
    \vspace{0.2cm}

    % \hspace{1cm}
    % \url{./Makefile}   \hspace{1.1cm} форматирование кода (\textbf{make check})
    % \vspace{0.2cm}
\end{frame}

%------------------------------------------------------------
\begin{frame}
\begin{itemize}
    \item баллы за модуль
    $$
        \left\lceil10\cdot\frac{t}{T}\right\rceil
    $$
    \begin{align*}
        T\quad  &\mbox{-- общее число домашек в модуле} \\
        t\quad  &\mbox{-- число зачтенных домашек в модуле}
    \end{align*}
    \item срок сдачи домашек -- две недели
    \item \url{./homeworks/homework_00.py} пример оформления домашки
    \item как сдавать домашки
    \begin{itemize}
        \item[] github private repo + add collaborator \textbf{alextanch}
        \item[] branch + pull request
        \item[] code review + approve
        \item[] interview + merge and close pull request
    \end{itemize}
    \item форматирование кода
    \begin{itemize}
        \item[] \url{isort} \url{homeworks}
	    \item[] \url{ruff} \url{check} \url{homeworks}
	    \item[] \url{ruff} \url{format} \url{homeworks}
    \end{itemize}
    
\end{itemize}
\end{frame}

\end{document}