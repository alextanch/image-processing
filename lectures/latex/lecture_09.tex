\documentclass[12pt, usepdftitle=false, aspectratio=1610]{beamer}

\usetheme{Madrid}
\usefonttheme[]{serif}
\setbeamertemplate{caption}[numbered]
\setbeamertemplate{navigation symbols}{}

\usepackage[T2A]{fontenc}			
\usepackage[utf8]{inputenc}			
\usepackage[english,russian]{babel}

\usepackage{
    mathtext,
    minted,
    cmap,
    multirow,
    textcomp,
    graphicx,
    wrapfig,
    subfig,
    mathtools,
    gensymb,
    amsmath,
    hyperref,
    wrapfig,
    tikz, tikz-3dplot, xcolor, physics, bm
} 
\usepackage[absolute, overlay]{textpos}
\usepackage[font=small,labelfont=bf]{caption}
\usepackage{subcaption}
\usetikzlibrary{calc,arrows.meta,positioning,backgrounds}

\setcounter{MaxMatrixCols}{12}

%\DeclarePairedDelimiter{\norm}{\lVert}{\rVert} 

\DeclareMathOperator*{\argmax}{arg\,max}
\DeclareMathOperator*{\argmin}{arg\,min}

\graphicspath{{./figs/09}}

\title[Лекция 9]{Проективные преобразования изображений}

\author{Александр Танченко!}
\institute{}
\date{2025}

\begin{document}

%------------------------------------------------------------
\begin{frame}
\titlepage
\end{frame}

%------------------------------------------------------------
\begin{frame}
\frametitle{Проективные преобразования изображений}
\begin{figure}[h]
    \includegraphics[height=0.4\textheight]{homography.jpg}
\end{figure}
\begin{itemize}
    \item Для каких точек на изображении $\mathbf{x}=(x\,, y)^\top$ 
          и соответствующих им точек на другом изображении $\mathbf{x}'=(x'\,, y')^\top$
          выполняется соотношение:
    \[
        \lambda\cdot
        \begin{bmatrix}
            x' \\ y' \\ 1
        \end{bmatrix}
        =
        \begin{bmatrix}
            h_{11} & h_{12} & h_{13} \\
            h_{21} & h_{22} & h_{23} \\
            h_{31} & h_{32} & h_{33}
        \end{bmatrix}
        \begin{bmatrix}
            x \\ y \\ 1
        \end{bmatrix}
    \]
    \item Такие преобразования называются \textbf{проективными преобразованиями} или \textbf{гомографиями}.
\end{itemize}
\end{frame}

%------------------------------------------------------------
\begin{frame}
\frametitle{Ортогональная проекция плоской сцены}
\begin{figure}[h]
    \includegraphics[height=0.3\textheight]{sim_transform.pdf}
\end{figure}
\begin{itemize}
    \item Рассмотрим случай, когда сцена является плоской, и оптическая ось камеры перпендикулярна плоскости сцены.
    \item Выберем начало мировой системы координат на плоскости сцены, а за ось $w$ возьмем оптическую ось камеры.
    \item Внешние параметры камеры:
    \[
        \mathbf{R} =
        \begin{bmatrix}
            \cos\theta & -\sin\theta & 0 \\
            \sin\theta & \cos\theta & 0 \\
            0 & 0 & 1
        \end{bmatrix},\quad
        \mathbf{t} =
        \begin{bmatrix}
            t_1 \\ t_2 \\ d
        \end{bmatrix}
    \]
    $d$ -- расстояние от оптического центра камеры до плоскости сцены.
\end{itemize}
\end{frame}

%------------------------------------------------------------
\begin{frame}
\frametitle{Ортогональная проекция плоской сцены}
Из модели камеры, при $f_x=f_y=f$ и $s=0$, получаем:
\[
    \lambda\cdot
    \begin{bmatrix}
        x \\ y \\ 1
    \end{bmatrix}
    =
    \begin{bmatrix}
        f & 0 & 0 \\
        0 & f & 0 \\
        0 & 0 & 1
    \end{bmatrix}
    \begin{bmatrix}
        \cos\theta & -\sin\theta & 0 & t_1 \\
        \sin\theta & \cos\theta & 0 & t_2 \\
        0 & 0 & 1 & d
    \end{bmatrix}
    \begin{bmatrix}
        u \\ v \\ 0 \\ 1
    \end{bmatrix}
\]
отсюда следует:
\[
    \begin{bmatrix}
        x \\ y \\ 1
    \end{bmatrix}
    =
    \begin{bmatrix}
        \rho\cos\theta & -\rho\sin\theta & t_x \\
        \rho\sin\theta & \rho\cos\theta & t_y \\
        0 & 0 & 1
    \end{bmatrix}
    \begin{bmatrix}
        u \\ v \\ 1
    \end{bmatrix}
\]
где 
$$
    \rho = \frac{f}{d},\quad
    t_x = \frac{f\cdot t_1}{d},\quad
    t_y = \frac{f\cdot t_2}{d}
$$
\end{frame}

%------------------------------------------------------------
\begin{frame}
\frametitle{Преобразование подобия (similarity transformation)}
\begin{figure}[h]
    \includegraphics[height=0.25\textheight]{sim.png}
\end{figure}
\begin{itemize}
    \item Таким образом, ортогональная проекция плоской сцены может быть описана преобразованием:
    \[
        \begin{bmatrix}
            x \\ y \\ 1
        \end{bmatrix}
        =
        \begin{bmatrix}
            \rho\cos\theta & -\rho\sin\theta & t_x \\
            \rho\sin\theta & \rho\cos\theta & t_y \\
            0 & 0 & 1
        \end{bmatrix}
        \begin{bmatrix}
            u \\ v \\ 1
        \end{bmatrix}
    \]
    или короче:
    \[
        \mathbf{x} = \mathbf{Sim}\left[\mathbf{w}\right]
    \]
    \item Такое преобразование называется \textbf{преобразованием подобия}.
    \item Преобразование подобия сохраняет углы между прямыми и зависит от 4 параметров.
\end{itemize}
\end{frame}

%------------------------------------------------------------
\begin{frame}
\frametitle{Аффинные преобразования изображений}
\begin{figure}[h]
    \includegraphics[height=0.3\textheight]{aff1.jpg}
    \hspace{1cm}
    \includegraphics[height=0.3\textheight]{aff2.jpg}
\end{figure}
\begin{itemize}
    \item Если угол наклона плоскости сцены небольшой, то связь между координатами точек на изображении и координатами точек на плоскости сцены можно аппроксимировать \textbf{аффинным преобразованием}:
    \[
        \begin{bmatrix}
            x \\ y \\ 1
        \end{bmatrix}
        =   
        \begin{bmatrix}
            a_{11} & a_{12} & t_x \\
            a_{21} & a_{22} & t_y \\
            0 & 0 & 1
        \end{bmatrix}
        \begin{bmatrix}
            u \\ v \\ 1
        \end{bmatrix}\,,
        \qquad
        \mathbf{x} = \mathbf{Aff}\left[\mathbf{w}\right]
    \]
    \item Аффинные преобразования сохраняют параллельность прямых и зависит от 6 параметров.
\end{itemize}
\end{frame}


%------------------------------------------------------------
\begin{frame}
\frametitle{Проективные преобразования изображений}
\begin{figure}[h]
    \includegraphics[height=0.3\textheight]{hom1.jpg}
    \hspace{1cm}
    \includegraphics[height=0.3\textheight]{hom2.jpg}
\end{figure}
\begin{itemize}
    \item В общем случае связь между координатами точек на изображении 
          и координатами точек на плоскости сцены описывается \textbf{проективным преобразованием (гомографией)}:
        \[
            \lambda\cdot
            \begin{bmatrix}
                x \\ y \\ 1
            \end{bmatrix}
            =   
            \begin{bmatrix}
                h_{11} & h_{12} & h_{13} \\
                h_{21} & h_{22} & h_{23} \\
                h_{31} & h_{32} & h_{33}
            \end{bmatrix}
            \begin{bmatrix}
                u \\ v \\ 1
            \end{bmatrix}\,,
            \qquad
            \mathbf{x} = \mathbf{Hom}\left[\mathbf{w}\right]
        \]
    \item Гомография отображает прямые в прямые и зависит от 8 параметров.
\end{itemize}
\end{frame}

%------------------------------------------------------------
\begin{frame}
\frametitle{Виды проективных преобразований}
\begin{figure}[h]
    \includegraphics[height=0.25\textheight]{transforms.pdf}
\end{figure}
\begin{itemize}
    \item Преобразование подобия является частным случаем аффинного преобразования.
    \[
        \mathbf{x} = \mathbf{Sim}\left[\mathbf{w}\right]\,,
        \qquad
        \mathrm{dof}\left(\mathbf{Sim}\right) = 4  
    \]
    \item Аффинное преобразование является частным случаем проективного преобразования.
    \[
        \mathbf{x} = \mathbf{Aff}\left[\mathbf{w}\right]\,,
        \qquad
        \mathrm{dof}\left(\mathbf{Aff}\right) = 6  
    \]
    \item Проективное преобразование является наиболее общим из рассмотренных преобразований.
    \[
        \mathbf{x} = \mathbf{Hom}\left[\mathbf{w}\right]\,,
        \qquad
        \mathrm{dof}\left(\mathbf{Hom}\right) = 8  
    \]
\end{itemize}
\end{frame}

%------------------------------------------------------------
\begin{frame}
\frametitle{Преобразование между изображениями плоской сцены}
\begin{figure}[h]
    \includegraphics[height=0.4\textheight]{composition.pdf}
    \hspace{1cm}
    \includegraphics[height=0.4\textheight]{satellite1.jpg}
    \includegraphics[height=0.3\textheight]{satellite2.jpg}
\end{figure}
Преобразование между двумя изображениями плоской сцены такое же как и преобразование между изображением и самой плоскостью сцены.
    \[
        \mathbf{x} = \mathbf{T_1}[\mathbf{w}]\,,\quad
        \mathbf{x}' = \mathbf{T_2}[\mathbf{w}]\,,
        \qquad
        \Rightarrow
        \quad
        \mathbf{x}' = \mathbf{T_2}\left[\mathbf{T_1^{-1}}[\mathbf{x}]\right]= \mathbf{T_3}[\mathbf{x}]
    \]
\end{frame}

%------------------------------------------------------------
\begin{frame}
\frametitle{Гомография между изображениями при повороте камеры}
\begin{figure}[h]
    \includegraphics[height=0.4\textheight]{rotation.pdf}
    \hspace{1cm}
    \includegraphics[height=0.3\textheight]{hom1.pdf}
    \includegraphics[height=0.3\textheight]{hom2.pdf}
\end{figure}
Если камера поворачивается вокруг своего оптического центра и сцена не плоская, 
то связь между координатами точек на двух изображениях также описывается гомографией:
    \[
        \mathbf{x}' = \mathbf{Hom}[\mathbf{x}]   
    \]
\end{frame}

%------------------------------------------------------------
\begin{frame}
\frametitle{Оценка параметров преобразования изображений}
\begin{figure}[h]
    \includegraphics[height=0.25\textheight]{learn_transform1.pdf}
    \includegraphics[height=0.25\textheight]{learn_transform2.pdf}
\end{figure}
Если известны соответствующие точки на двух изображениях $\left\{\mathbf{x}_i\right\}$,
$\left\{\mathbf{x}'_i\right\}$ и координаты этих точек связаны преобразованием $\mathbf{x}'_i=\mathbf{T}[\mathbf{x}_i\,,\boldsymbol{\theta}]$, 
то параметры этого преобразования можно оценить методом максимального правдоподобия:
\[
    \hat{\boldsymbol{\theta}} = \argmin_{\boldsymbol{\theta}} \left(-\sum_{i} 
    \log\left[\mathrm{Norm}_{\mathbf{x}'_i}\left[\mathbf{T}[\mathbf{x}_i\,, \boldsymbol{\theta}]\,,\sigma^2\mathbf{I}\right]\right]
    \right)
\]
Подставляя формулу нормального распределения, получаем задачу наименьших квадратов:
\[
    \hat{\boldsymbol{\theta}} = \argmin_{\boldsymbol{\theta}} \sum_{i} 
    \norm{\mathbf{x}'_i - \mathbf{T}[\mathbf{x}_i\,, \boldsymbol{\theta}]}^2
\]
\end{frame}

%------------------------------------------------------------
\begin{frame}
\frametitle{Оценка параметров преобразования подобия}
\begin{itemize}
    \item Для преобразования подобия
    \[
        \mathbf{T}[\mathbf{x}]=\mathbf{Sim}[\mathbf{x}]=
        \rho\cdot\boldsymbol{\Omega}\mathbf{x} + \boldsymbol{\tau}=
        \rho\cdot\begin{bmatrix}
            \cos\theta & -\sin\theta \\
            \sin\theta &  \cos\theta
        \end{bmatrix}\mathbf{x} +
        \begin{bmatrix}
            t_x \\ t_y
        \end{bmatrix}
    \]
    \item Получим следующую задачу наименьших квадратов:
    \[
        \widehat{\boldsymbol{\Omega}}\,, \hat{\rho}\,, \hat{\boldsymbol{\tau}} =
        \argmin_{\boldsymbol{\Omega}\,, \rho\,, \boldsymbol{\tau}} 
        \sum_{i=1}^{N} 
        \norm{\mathbf{x}'_i - \left(\rho\cdot\boldsymbol{\Omega}\mathbf{x}_i + \boldsymbol{\tau}\right)}^2
    \]
    \item Число параметров: $ \mathrm{dof}(\mathbf{Sim}) = 4 $, 
    поэтому необходимо минимум две соответствующие точки ($N \geq 2$).
    \item Эту задачу можно решить аналитически.
\end{itemize}

\end{frame}

%------------------------------------------------------------
\begin{frame}
\frametitle{Оценка параметров преобразования подобия}
Пусть
\[
    \boldsymbol{\mu} = \frac{1}{N}\sum_{i=1}^{N}\mathbf{x}_i\,,\quad
    \boldsymbol{\mu}' = \frac{1}{N}\sum_{i=1}^{N}\mathbf{x}'_i
\]
$$
    \mathbf{A}=[\mathbf{x}_1 - \boldsymbol{\mu}\,, \mathbf{x}_2 - \boldsymbol{\mu}\,, \ldots\,, \mathbf{x}_N - \boldsymbol{\mu}] \,,\quad
    \mathbf{B}=[\mathbf{x}'_1 - \boldsymbol{\mu}'\,, \mathbf{x}'_2 - \boldsymbol{\mu}'\,, \ldots\,, \mathbf{x}'_N - \boldsymbol{\mu}']
$$
SVD-разложение матрицы $\mathbf{B}\mathbf{A}^T$:
\[
    \mathbf{B}\mathbf{A}^\top = \mathbf{U}\mathbf{L}\mathbf{V}^\top
\]
Тогда
\[
    \widehat{\boldsymbol{\Omega}}=\mathbf{V}\mathbf{U}^T
\]
\[
    \widehat{\rho}=
    \frac{\sum_{i=1}^N(\mathbf{x}'_i-\boldsymbol{\mu}')^\top\widehat{\boldsymbol{\Omega}}(\mathbf{x}_i-\boldsymbol{\mu})}
    {\sum_{i=1}^N\norm{\mathbf{x}_i-\boldsymbol{\mu}}^2}
\]
\[
    \widehat{\boldsymbol{\tau}} = \boldsymbol{\mu}' - \widehat{\rho}\cdot\widehat{\boldsymbol{\Omega}}\boldsymbol{\mu}
\]
\end{frame}

%------------------------------------------------------------
\begin{frame}
\frametitle{Оценка параметров проективного преобразования}
\begin{itemize}
    \item Для проективного преобразования $\mathbf{x}'_i=\mathbf{Hom}[\mathbf{x}_i]$
    \[
        \lambda_i\cdot
        \begin{bmatrix}
            x'_i \\ y'_i \\ 1
        \end{bmatrix}
        = \mathbf{H}\cdot\mathbf{x}_i =
        \begin{bmatrix}
            h_{11} & h_{12} & h_{13} \\
            h_{21} & h_{22} & h_{23} \\
            h_{31} & h_{32} & h_{33}
        \end{bmatrix}
        \begin{bmatrix}
            x_i \\ y_i \\ 1
        \end{bmatrix}
    \]
    \item Отсюда
    \begin{align*}
        x'_i &= \frac{h_{11}x_i + h_{12}y_i + h_{13}}{h_{31}x_i + h_{32}y_i + h_{33}} \\
        y'_i &= \frac{h_{21}x_i + h_{22}y_i + h_{23}}{h_{31}x_i + h_{32}y_i + h_{33}}
    \end{align*}
\end{itemize}
\end{frame}

%------------------------------------------------------------
\begin{frame}
\frametitle{Оценка параметров проективного преобразования}
\begin{itemize}
    \item Подставляя эти выражения в задачу наименьших квадратов, получаем:
    \[
        \widehat{\mathbf{H}} =
        \argmin_{\mathbf{H}} \sum_{i=1}^{N}
        \mathrm{dist}^2(\mathbf{x}'_i\,, \mathbf{x}_i)
    \]
    где
    \[
        \mathrm{dist}^2(\mathbf{x}'_i\,, \mathbf{x}_i) =
        \left(x'_i-\frac{h_{11}x_i + h_{12}y_i + h_{13}}{h_{31}x_i + h_{32}y_i + h_{33}}\right)^2 +
        \left(y'_i-\frac{h_{21}x_i + h_{22}y_i + h_{23}}{h_{31}x_i + h_{32}y_i + h_{33}}\right)^2
    \]
    --- \textbf{ошибка репроекции}.
    \item Число параметров: $ \mathrm{dof}(\mathbf{Hom}) = 8 $, 
    поэтому необходимо минимум четыре соответствующие точки ($N \geq 4$).
    \item Эту задачу нелинейной оптимизации можно решить численными методами,
    например методом Левенберга-Марквардта. 
    \item Построим начальное приближение решения методом DLT.
\end{itemize}
\end{frame}

%------------------------------------------------------------
\begin{frame}
\frametitle{Оценка параметров проективного преобразования}
\[
        \lambda_i\cdot
        \begin{bmatrix}
            x'_i \\ y'_i \\ 1
        \end{bmatrix} =
        \begin{bmatrix}
            h_{11} & h_{12} & h_{13} \\
            h_{21} & h_{22} & h_{23} \\
            h_{31} & h_{32} & h_{33}
        \end{bmatrix}\cdot
        \begin{bmatrix}
            x_i \\ y_i \\ 1
        \end{bmatrix}
\]
\begin{itemize}
    \item Умножим эту формулу векторно на 
    $\begin{bmatrix}
        x'_i \\ y'_i \\ 1
    \end{bmatrix}$:
    \[
        \mathbf{0} =
        \begin{bmatrix}
            x'_i \\ y'_i \\ 1
        \end{bmatrix} \times
        \begin{bmatrix}
            h_{11} & h_{12} & h_{13} \\
            h_{21} & h_{22} & h_{23} \\
            h_{31} & h_{32} & h_{33}
        \end{bmatrix}\cdot
        \begin{bmatrix}
            x_i \\ y_i \\ 1
        \end{bmatrix}
    \]
    \item Два из этих трех уравнений линейно независимы:
    \[
        \begin{bmatrix}
            0 & 0 & 0 & -x_i & -y_i & -1 & y'_i x_i & y'_i y_i & y'_i \\
            x_i & y_i & 1 & 0 & 0 & 0 & -x'_i x_i & -x'_i y_i & -x'_i
        \end{bmatrix}
        \cdot\boldsymbol{\theta}=\mathbf{0}
    \]
    где
    \[
        \boldsymbol{\theta} =
        [h_{11}\,, h_{12}\,, h_{13}\,, h_{21}\,, h_{22}\,, h_{23}\,, h_{31}\,, h_{32}\,, h_{33}]^\top
    \]
\end{itemize}
\end{frame}

%------------------------------------------------------------
\begin{frame}
\frametitle{Оценка параметров проективного преобразования}
\begin{itemize}
    \item Для $N$ соответствующих точек получаем систему уравнений:
    \[
        \mathbf{A}\cdot\boldsymbol{\theta}=\mathbf{0}
    \]
    где матрица $\mathbf{A}$ размера $2N\times 9$ составляется из блоков:
    \[
        \begin{bmatrix}
            0 & 0 & 0 & -x_i & -y_i & -1 & y'_i x_i & y'_i y_i & y'_i \\
            x_i & y_i & 1 & 0 & 0 & 0 & -x'_i x_i & -x'_i y_i & -x'_i
        \end{bmatrix}
    \]
    \item Оценка для $\boldsymbol{\theta}$ находится как решение задачи:
    \[
        \widehat{\boldsymbol{\theta}} = \argmin_{\boldsymbol{\theta}} \norm{\mathbf{A}\cdot\boldsymbol{\theta}}^2
    \quad \text{при условии} \quad \norm{\boldsymbol{\theta}}=1
    \]
    \item Такое решение находится как правый столбец матрицы $\mathbf{V}$ в SVD-разложении
    \[
        \mathbf{A} = \mathbf{U}\mathbf{L}\mathbf{V}^\top
    \]
\end{itemize}
\end{frame}

%------------------------------------------------------------
\begin{frame}
\frametitle{Алгоритм RANSAC (RANdom SAmple Consensus)}
\begin{figure}[h]
    \includegraphics[height=0.5\textheight]{ransac1.jpg}
\end{figure}
\begin{itemize}
    \item Пары соответствующих точек состоят из inliers и outliers.
    \item Перед оценкой параметров гомографии необходимо отфильтровать outliers.
    \item C помощью RANSAC можно отфильтровать outliers и оценить параметры гомографии только на inliers.
\end{itemize}
\end{frame}

%------------------------------------------------------------
\begin{frame}
\frametitle{Алгоритм RANSAC для оценки гомографии}
\begin{itemize}
    \item[1.] \textbf{robust estimation}: повторить $N$ раз
    \begin{itemize}
        \item[(a)] выбрать 4 случайных соответствующих точек $\mathbf{x}_i\leftrightarrow\mathbf{x}'_i$
                   и вычислить $\mathbf{H}$ методом DLT
        \item[(b)] разделить точки на inliers и outliers по порогу $T$ и ошибке репроекции:
                   $$
                    \mathrm{dist}^2(\mathbf{x}'_i\,, \mathbf{x}_i)<T
                   $$
    \end{itemize} 
    \item[2.] выбрать $\mathbf{H}$ c наибольшим числом inliers в качестве начальной оценки алгоритма Левенберга-Марквардта
    \item[3.] \textbf{optimal estimation}:  методом Левенберга-Марквардта
          вычислить финальную гомографию $\mathbf{H}$ только на inliers
\end{itemize}
Число итераций $N$ и порог $T$ выбираются адаптивно.
\end{frame}

%------------------------------------------------------------
\begin{frame}
\frametitle{Пример алгоритма RANSAC для оценки гомографии}
\begin{figure}[h]
    \includegraphics[height=0.5\textheight]{ransac1.jpg}
    \includegraphics[height=0.5\textheight]{ransac2.jpg}
\end{figure}
\begin{itemize}
    \item (a), (b) -- исходные изображения
    \item (c), (d) -- найденные соответствия ключевых точек
    \item (e), (f) -- inliers, найденные RANSAC 
\end{itemize}
\end{frame}

%------------------------------------------------------------
\begin{frame}
\frametitle{Построение панорамного изображения}
\begin{figure}[h]
    \includegraphics[height=0.5\textheight]{panorama.jpg}
\end{figure}
\begin{itemize}
    \item Три изображения сцены, для которых камера поворачивается вокруг своего оптического центра.
    \item Для каждого изображения построены пять соответствующих ключевых точек.
    \item Панорама строится при помощи гомографий, преобразующих первое и третье изображения в систему координат второго изображения.
\end{itemize}
\end{frame}

%------------------------------------------------------------
\end{document}

