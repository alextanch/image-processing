\documentclass[12pt, usepdftitle=false, aspectratio=1610]{beamer}

\usetheme{Madrid}

\usefonttheme[]{serif}
\setbeamertemplate{caption}[numbered]
\setbeamertemplate{navigation symbols}{}

\usepackage[T2A]{fontenc}			
\usepackage[utf8]{inputenc}			
\usepackage[english,russian]{babel}

\usepackage{
    mathtext,
    minted,
    cmap,
    multirow,
    textcomp,
    graphicx,
    wrapfig,
    subfig,
    mathtools,
    gensymb,
    amsmath,
    hyperref
}
\usepackage[font=small,labelfont=bf]{caption}
\usepackage[absolute, overlay]{textpos}

\DeclarePairedDelimiter{\norm}{\lVert}{\rVert} 
\DeclareMathOperator*{\argmax}{arg\,max}
\DeclareMathOperator*{\argmin}{arg\,min}

\graphicspath{{./figs/05}}

\title[Лекция 5]{Детекция и сопоставление ключевых точек на изображениях}

\author{Александр Танченко}
\institute{}
\date{2025}

\begin{document}

%------------------------------------------------------------
\begin{frame}
    \titlepage
\end{frame}

%------------------------------------------------------------
\begin{frame}
\frametitle{Детекция и сопоставление ключевых точек}
\begin{figure}
    \centering
    \includegraphics[height=0.6\textheight]{two_views.jpg}
\end{figure}
\end{frame}

%------------------------------------------------------------
\begin{frame}
\frametitle{Детекция и сопоставление ключевых точек}
\begin{figure}
    \centering
    \includegraphics[height=0.6\textheight]{matches.jpg}
\end{figure}
\end{frame}

%------------------------------------------------------------
\begin{frame}
\frametitle{Детекция и сопоставление ключевых точек}
\begin{figure}
    \centering
    \includegraphics[height=0.6\textheight]{scene3d.pdf}
    \hspace{1cm}
    \includegraphics[height=0.6\textheight]{two_cameras.pdf}
\end{figure}
\end{frame}

%------------------------------------------------------------
\begin{frame}
\frametitle{Детекция ключевых точек (keypoints detection)}
\begin{figure}
    \centering
    \includegraphics[width=0.5\textwidth]{two_views.png}
\end{figure}
\begin{figure}
    \centering
    \includegraphics[height=0.3\textheight]{keypoints.png}
\end{figure}
\begin{itemize}
    \item ``плоские'' точки и точки на границах не являются ключевыми
    \item угловые точки -- хорошие кандидаты в ключевые точки
\end{itemize}
\end{frame}

%------------------------------------------------------------
\begin{frame}
\frametitle{Сопоставление ключевых точек (keypoints matching)}
\begin{figure}
    \centering
    \includegraphics[height=0.45\textheight]{matches.jpg}
\end{figure}
\begin{itemize}
    \item для ключевых точек найти дескрипторы, устойчивые при геометрических и цветовых аугментациях изображений
    $$
        \left\{x_i\,,y_i\,,\mathbf{d}_i\right\}\qquad
        \left\{x'_j\,,y'_j\,,\mathbf{d}'_j\right\}\qquad
        \mathbf{d}_i\,,\mathbf{d}'_j\in\mathbf{R}^D
    $$
    \item сопоставить ключевые точки по близости дескрипторов в пространстве $\mathbf{R}^D$
    $$
        \mathbf{d}_i\approx\mathbf{d}'_j
        \quad\Rightarrow\quad
        (x_i\,,y_i)\leftrightarrow (x'_j\,,y'_j)
    $$
\end{itemize}
\end{frame}

%------------------------------------------------------------
\begin{frame}
\frametitle{Детекторы ключевых точек и их дескрипторов}
\begin{itemize}
    \item Классические детекторы
    \begin{itemize}
        \item ORB (Oriented FAST and Rotated BRIEF)
        \item SURF (Speeded-Up Robust Features)
        \item \textbf{SIFT} (Scale Invariant Feature Transform)
        
        \href{https://www.cs.ubc.ca/~lowe/papers/ijcv04.pdf}
        {D.G. Lowe <<Distinctive Image Features from Scale-Invariant Keypoints>>, 2004}
    \end{itemize}
    \item Нейросетевые детекторы
    \begin{itemize}
        \item D2-Net (\url{https://github.com/mihaidusmanu/d2-net})
        \item LoFTR (\url{https://github.com/zju3dv/LoFTR})
        \item \textbf{SuperPoint} 
        
        \href{https://arxiv.org/abs/1712.07629}
        {D. DeTone etc. <<Self-Supervised Interest Point Detection and Description>>, 2017}
    \end{itemize}
\end{itemize}

\end{frame}

%------------------------------------------------------------
\begin{frame}
\frametitle{SIFT: построение пирамиды Гаусса}
\begin{textblock}{8}(0.5, 3)
    Требуется устойчивость относительно изменения масштаба изображения.
    \vspace{0.5cm}

    Разложим изображение в гауссовскую пирамиду разбитую на октавы.
    $$
        L_\sigma = G_\sigma \circ f
    $$
    $$
        G_\sigma=K\cdot\exp\left\{-\frac{s^2+t^2}{2\sigma^2}\right\}
    $$
\end{textblock}
\begin{textblock}{5}(8.5, 2)
    \includegraphics[height=0.85\textheight]{pyramid.pdf}
\end{textblock}
\end{frame}

%------------------------------------------------------------
\begin{frame}
\frametitle{SIFT: построение DoG пирамиды}
\begin{figure}
    \includegraphics[height=0.5\textheight]{log_pyramid.pdf}
\end{figure}
DoG (Difference of Gaussians) пирамида
$$
     D(x,y,\sigma) = L_{k\sigma}(x,y)-L_{\sigma}(x,y)  
$$
Можно показать, что
$$
    D(x,y,\sigma)\approx
    \Delta \left(G_\sigma\circ f\right)=\mathrm{LoG}
$$
\end{frame}

%------------------------------------------------------------
\begin{frame}
\frametitle{SIFT: локальные экстремумы в пирамиде DoG}
\begin{figure}
    \includegraphics[height=0.4\textheight]{extremum.pdf}
\end{figure}
кандидаты в ключевые точки -- локальные экстремумы функции трех переменных $D(x,y,\sigma)$
в окрестности $3\times3$
$$
    \left\{x_\ast,y_\ast,\sigma_\ast\right\}=
    \left\{\argmin_{x,y,\sigma} D(x,y,\sigma)\right\}
    \cup
    \left\{\argmax_{x,y,\sigma} D(x,y,\sigma)\right\}
$$
\end{frame}

%------------------------------------------------------------
\begin{frame}
\frametitle{SIFT: построение ключевых точек}
\begin{figure}
    \includegraphics[height=0.35\textheight]{building.png}
    \hspace{1cm}
    \includegraphics[height=0.35\textheight]{sift_keypoints.pdf}
\end{figure}
\begin{itemize}
    \item найденные экстремумы содержат граничные и угловые точки
    \item вычислим собственные  числа матрицы Гессе в точках $\left\{x_\ast,y_\ast,\sigma_\ast\right\}$
    $$
        \mathbf{H}=
        \begin{bmatrix}
        D_{xx} & D_{xy} \\
        D_{xy} & D_{yy}
        \end{bmatrix}\,,
        \qquad
        \mathbf{H}\cdot\mathbf{v}_1=\lambda_1\cdot\mathbf{v}_1\,,
        \quad
        \mathbf{H}\cdot\mathbf{v}_2=\lambda_2\cdot\mathbf{v}_2
    $$
    \item ключевые точки находятся из условия 

  $$
    \frac{(\lambda_1+\lambda_2)^2}{\lambda_1\lambda_2} < 
    \frac{(r+1)^2}{r}
    \qquad
    \left(r = 10\,,\quad \frac{(r+1)^2}{r}=12.1\right)
  $$
\end{itemize}
\end{frame}

%------------------------------------------------------------
\begin{frame}
\frametitle{SIFT: вычисление ориентаций ключевых точек}
\begin{figure}
    \includegraphics[height=0.3\textheight]{orientation1.pdf}
    \hspace*{1cm}
    \includegraphics[height=0.3\textheight]{orientation2.jpg}
\end{figure}
\begin{itemize}
    \item для окрестности каждой особой точки в пирамиде Лапласа $L(x,y)=L_{\sigma_\ast}(x,y)$
    вычислим  модуль и направление градиента
    $$
        m=\sqrt{(L'_x)^2+(L'_y)^2}\qquad
        \alpha=\arctan\left(\frac{L'_x}{L'_y}\right)
    $$
    \item построим гистограмму направлений градиентов, вклад в гистограмму умножается
    на $m$
    \item в качестве ориентации ключевой точки возьмем пики на гистограмме
\end{itemize}
\end{frame}

%------------------------------------------------------------
\begin{frame}
\frametitle{SIFT: построение дескрипторов ключевых точек}
\begin{textblock}{8}(-0.1, 3)
    \begin{itemize}
        \item для устойчивости к повороту градиенты поворачиваются на угол ориентации ключевой точки
        \item дескриптор - это вектор, состоящий из  $16\cdot8=128$  значений гистограмм
        $$
            \mathbf{d}\in\mathbf{R}^{128}
        $$
    \end{itemize}
\end{textblock}
\begin{textblock}{5}(7.8, 3)
    \includegraphics[height=0.75\textheight]{sift_descriptor.pdf}
\end{textblock}
\end{frame}

%------------------------------------------------------------
\begin{frame}[fragile]
\frametitle{SIFT: устойчивость к цветовой аугментации изображения}
\begin{itemize}
\item дескриптор нормируется для устойчивости к  цветовой аугментации
 $$
    \mathbf{d}=\frac{\mathbf{d}}{\norm{\mathbf{d}}}
$$
\item в OpenCV реализации
\vspace*{0.4cm}

\begin{figure}[h!]
    \centering
    \begin{minted}{python}
        d = np.clip(512 * d, 0, 255)
    \end{minted}
\end{figure}
\end{itemize}
\end{frame}

%------------------------------------------------------------
\begin{frame}
\frametitle{SuperPoint: архитектура нейронной сети}
\begin{figure}
    \includegraphics[height=0.6\textheight]{superpoint.pdf}
\end{figure}
$$
    \mbox{\textbf{Grayscale Image}}
    \mapsto
    \mbox{\textbf{Superpoint Network}}
    \mapsto
    \left\{x_n, y_y, \mathbf{d}_n\right\}
    \quad
$$
$$
    \mathbf{d}_n\in\mathbf{R}^{256}\qquad
    \norm{\mathbf{d}_n}=1
$$
\end{frame}

%------------------------------------------------------------
\begin{frame}[fragile]
\frametitle{SuperPoint: обучение нейронной сети}
\begin{figure}
    \includegraphics[height=0.6\textheight]{superpoint_train.png}
\end{figure}
Устойчивость дескрипторов достигается с помощью цветовой и геометрической
аугментации изображений во время обучения.
\end{frame}

%------------------------------------------------------------
\begin{frame}[fragile]
\frametitle{SuperPoint: вычисление расстояний между дескрипторами}
\begin{itemize}
    \item норма SuperPoint дескрипторов равна единице
    $$
        \left\{\mathbf{d}_n\right\}_{n=1}^N
        \quad
        \left\{\mathbf{d}'_m\right\}_{m=1}^M
        \qquad
        \norm{\mathbf{d}_n}=\norm{\mathbf{d}'_m}=1
    $$
    \item расстояния можно вычислить с помощью скалярных произведений
    $$
        \norm{\mathbf{d}'_m-\mathbf{d}_n}^2=
        \norm{\mathbf{d}'_m}^2+\norm{\mathbf{d}_n}^2-2\cdot(\mathbf{d}'_m\,,\mathbf{d}_n)=
        2\cdot[1-(\mathbf{d}'_m\,,\mathbf{d}_n)]
    $$
    \item скалярные произведения вычисляются с помощью матричного умножения
    $$
        \boldsymbol{D}=
        \begin{bmatrix*}[c]
            \mathbf{d}_1 \\ \mathbf{d}_2 \\ \vdots \\ \mathbf{d}_N 
        \end{bmatrix*}
        \in\mathcal{M}(N\times256)
        \qquad
        \boldsymbol{D}'=
        \begin{bmatrix*}[c]
            \mathbf{d}'_1 \\ \mathbf{d}'_2 \\ \vdots \\ \mathbf{d}_M 
        \end{bmatrix*}
        \in\mathcal{M}(M\times256)
    $$
    $$
        \left[\boldsymbol{D}'\cdot \boldsymbol{D}^T\right]_{mn}=
        (\mathbf{d}'_m\,,\mathbf{d}_n)
    $$
\end{itemize}
\end{frame}

%------------------------------------------------------------
\begin{frame}
\frametitle{Ratio test matching}
\begin{figure}
    \includegraphics[height=0.4\textheight]{audi.pdf}
    %\caption*{$(x, y, \mathbf{d})$ - на левом изображении $\quad(x', y', \mathbf{d}')$ - на правом изображении}
\end{figure}
\begin{itemize}
    \item для каждого дескриптора $\mathbf{d}$ найдем два ближайших соседа $\mathbf{d}'_1$, $\mathbf{d}'_2$
    \item зададим порог $0<T<1$
    \item $(x,y)$, $(x'_1,y'_1)$ -- пара соответствующих точек, если
    $$
        \norm{\mathbf{d}-\mathbf{d}'_1} < T\cdot\norm{\mathbf{d}-\mathbf{d}'_2}
    $$
\end{itemize}
\end{frame}

%------------------------------------------------------------
\begin{frame}
\frametitle{Cross check matching}
\begin{figure}
    \includegraphics[height=0.3\textheight]{cross_check.png}
\end{figure}
$(x,y)$, $(x',y')$ -- пара соответствующих точек, если
\begin{itemize}
    \item
    $$
        \mathbf{d}'\quad\mbox{ближайший сосед}\quad\mathbf{d}
    $$
    \item
    $$
        \mathbf{d}\quad\mbox{ближайший сосед}\quad\mathbf{d}'
    $$
    \item
    $$
        \norm{\mathbf{d}-\mathbf{d}'} < T
    $$
\end{itemize}
\end{frame}

%------------------------------------------------------------
\begin{frame}[fragile]
\frametitle{OpenCV Matching}
\begin{itemize}
    \item Brute-Force -- алгоритм поиска ближайших соседей полным перебором (точный, но не быстрый)
    \vspace*{0.1cm}
    \begin{minted}{python}
        bf = cv2.BFMatcher(
            cv2.NORM_L2, 
            crossCheck=True
        )
    \end{minted}
    \vspace*{0.2cm}
    \item FLANN -- алгоритм поиска ближайших соседей c помощью нескольких kd-деревьев
    (быстрый, но не точный)
    \vspace*{0.1cm}
    \begin{minted}{python}
        flann = cv2.FlannBasedMatcher(
            {"algorithm": 1, "trees": 5}
        )
    \end{minted}
\end{itemize}
\end{frame}

\end{document}