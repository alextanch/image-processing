\documentclass[
    12pt, 
    usepdftitle=false,
    aspectratio=1610
]{beamer}

\usetheme{Madrid}

\usefonttheme[]{serif}
\setbeamertemplate{caption}[numbered]
\setbeamertemplate{navigation symbols}{}

\usepackage[T2A]{fontenc}			
\usepackage[utf8]{inputenc}			
\usepackage[english,russian]{babel}

\usepackage{
    mathtext,
    minted,
    cmap,
    multirow,
    textcomp,
    graphicx,
    wrapfig,
    subfig,
    mathtools,
    gensymb
}
\usepackage[font=small,labelfont=bf]{caption}

\DeclarePairedDelimiter{\norm}{\lVert}{\rVert} 

\graphicspath{{./figs/02}}

\title[Лекция 2]{
    Пространственная и частотная фильтрация изображений
}

\author{Александр Танченко}
\institute{}
\date{2025}

\begin{document}

%------------------------------------------------------------
\begin{frame}
    \titlepage
\end{frame}

%---------------------------------------------------------------------
\begin{frame}
    \frametitle{Фильтрация изображений}
    \begin{center}
        \includegraphics[height=0.7\textheight]{filters.png}
    \end{center}
    $$
        f\quad\rightarrow\quad\mbox{\textbf{фильтр}}\quad\rightarrow\quad g
    $$
\end{frame}

%---------------------------------------------------------------------
\begin{frame}
    \frametitle{Фильтрация в пространственной и частотной области}
    \begin{itemize}
        \item фильтрация в пространственной области
        \begin{center}
            \includegraphics[height=0.3\textheight]{gradient4.pdf}
        \end{center}
        \item фильтрация в частотной области
        \begin{center}
            \includegraphics[height=0.3\textheight]{freq_filter1.png}
            \includegraphics[height=0.3\textheight]{freq_filter2.png}
            \includegraphics[height=0.3\textheight]{freq_filter3.png}
            \includegraphics[height=0.3\textheight]{freq_filter4.png}
        \end{center}
    \end{itemize}
\end{frame}

%---------------------------------------------------------------------
\begin{frame}
\frametitle{Виды пространственной фильтрации}
    \begin{itemize}
        \item blur filtering
        \begin{center}
            \includegraphics[height=0.2\textheight]{blur.png}
        \end{center}
        \item sharp filtering
        \begin{center}
            \includegraphics[height=0.2\textheight]{sharp.png}
        \end{center}
        \item edge detection filtering
        \begin{center}
            \includegraphics[height=0.2\textheight]{edge.png}
        \end{center}
    \end{itemize}
\end{frame}

%---------------------------------------------------------------------
\begin{frame}
\frametitle{Виды фильтров в частотной области}
    \begin{itemize}
        \item фильтр низких частот (low-pass filter)
        \begin{center}
            \includegraphics[height=0.35\textheight]{lowpass.png}
        \end{center}
        \item фильтр высоких частот (high-pass filter)
        \begin{center}
            \includegraphics[height=0.35\textheight]{highpass.png}
        \end{center}
    \end{itemize}
\end{frame}

%---------------------------------------------------------------------
\begin{frame}
    \frametitle{$\blacksquare$ Image padding}
    \begin{center}
        \includegraphics[width=0.8\textwidth]{paddings.png}
    \end{center}
    При фильтрации в пространственной области используют image padding.
    \vspace{0.5cm}

    \begin{itemize}
        \item \textbf{zero padding}
        \item \textbf{circular padding}
        \item \textbf{reflect padding}
        \item \textbf{replicate padding}
    \end{itemize}
\end{frame}

%---------------------------------------------------------------------
\begin{frame}
    \frametitle{Корреляция ядра с изображением}
    \url{https://github.com/alextanch/image-processing/blob/main/data/correlation.gif}
    \begin{center}
        \includegraphics[height=0.3\textheight]{kernel.pdf}
        \includegraphics[height=0.3\textheight]{correlation.pdf}
    \end{center}
    Корреляция ядра $w$ размера $n\times n$ с изображением $f$.
    $$
        (w\bullet f)(x,y)=\sum_{s=-k}^k\sum_{t=-k}^k w(s,t)f(x+s,y+t)
        \quad\mbox{число операций}\quad
        \mathcal{O}\left(H\cdot W\cdot n^2\right)
    $$
    $$
        n=2k+1
    $$
\end{frame}

%---------------------------------------------------------------------
\begin{frame}
    \frametitle{Свертка ядра с изображением}
    \begin{center}
        \includegraphics[height=0.3\textheight]{kernel.pdf}
    % \end{center}
    % \begin{center}
        \includegraphics[height=0.3\textheight]{convolution.pdf}
    \end{center}
    Свертка ядра с изображением: 
    $$
        w\circ f = (w \mbox{ повернуть на 180 градусов})\bullet f
    $$
    $$
        (w\circ f)(x,y)=\sum_{s=-k}^k\sum_{t=-k}^k w(s,t)f(x-s,y-t)
        \quad\mbox{число операций}\quad
        \mathcal{O}\left(H\cdot W\cdot n^2\right)
    $$
\end{frame}

%---------------------------------------------------------------------
\begin{frame}
    \frametitle{$\blacksquare$ Свойства корреляции и свертки}
    \begin{center}
        \begin{tabular}{ |l|c|c| } 
            \hline
            \textbf{Свойства} & \textbf{Свертка} & \textbf{Корреляция} \\
            \hline
            коммутативность  & $w\circ f=f\circ w$ & --- \\ 
            ассоциативность  & $w_2\circ (w_1\circ f) = (w_2\circ w_1)\circ f$ & --- \\ 
            дистрибутивность & $w\circ(f+g)=w\circ f + w\circ f$ & $w\bullet(f+g)=w\bullet f + w\bullet g$ \\ 
            \hline
        \end{tabular}
    \end{center}
    \begin{itemize}
        \item корреляция и свертка -- примеры линейных фильтров
        \item если ядро $w$ симметрично при повороте на $180^\circ$,
              то свертка совпадает с корреляцией
    \end{itemize}
\end{frame}

%---------------------------------------------------------------------
\begin{frame}
\frametitle{Сепарабельные свертки}
\begin{itemize}
    \item ядро свертки $w$ называется \textbf{сепарабельным}, если
    $$
        w = u \cdot v^t
    $$
    \item пример сепарабельной свертки
        $$
            w = 
            \begin{bmatrix*}[c]
                1 & 2 & 1\\
                2 & 4 & 2\\
                1 & 2 & 1
            \end{bmatrix*}=
            \begin{bmatrix*}[c]
                1 \\ 2\\ 1
            \end{bmatrix*}\cdot
            \begin{bmatrix*}[c]
                1 & 2 & 1
            \end{bmatrix*}
            \qquad\Rightarrow\qquad
            u =
            \begin{bmatrix*}[r]
                1 \\ 2\\ 1
            \end{bmatrix*}\,,\quad
            v =
            \begin{bmatrix*}[r]
                1 \\ 2\\ 1
            \end{bmatrix*}
        $$
\end{itemize}
\end{frame}

%---------------------------------------------------------------------
\begin{frame}
\frametitle{Сепарабельные свертки}
\begin{itemize}
    \item ядро $w$ -- сепарабельно, если $\mathrm{rank}(w)=1$ 
    \item как найти $u$ и $v$ в произведении
    $$
        w = u \cdot v^t
    $$
    \begin{enumerate}
        \item[(a)] найти ненулевой элемент $e\in w$
        \item[(b)] пусть $c$, $r$ -- столбец и строка, содержащие элемент $e$, тогда
        $$
            u = c\,,\qquad
            v = \frac{r^t}{e} 
        $$
    \end{enumerate}
    \item пример
        $$
            w = 
            \begin{bmatrix*}[r]
                1 & 2 & 1\\
                \mathbf{2} & 4 & 2\\
                1 & 2 & 1
            \end{bmatrix*}\,,\quad
            u =
            \begin{bmatrix*}[r]
                1 \\ 2\\ 1
            \end{bmatrix*}\,,\quad
            v = \frac{1}{2}
            \begin{bmatrix*}[r]
                2 \\ 4 \\ 2
            \end{bmatrix*}=
            \begin{bmatrix*}[r]
                1 \\ 2 \\ 1
            \end{bmatrix*}
        $$
\end{itemize}
\end{frame}

%---------------------------------------------------------------------
\begin{frame}
\frametitle{$\blacksquare$ Вычислительная эффективность сепарабельных сверток}
\begin{itemize}
    \item из коммутативности и ассоциативности свертки
    $$
        w\circ f = \left(u \cdot v^t\right)\circ f=u\circ v \circ f
    $$ 
    \item вычислительная эффективность
    $$
        \frac{\mathcal{O}\left(H\cdot W\cdot n^2\right)}{2\cdot\mathcal{O}\left(H\cdot W\cdot n\right)}=\mathcal{O}(n)
    $$
\end{itemize}
\end{frame}

%---------------------------------------------------------------------
% \begin{frame}
% \frametitle{$\blacksquare$ Template matching by correlation}
% \begin{figure}
%     \centering
%     \includegraphics[height=0.2\textheight]{cats.jpg}
%     \includegraphics[height=0.05\textheight]{template.jpg}
% \end{figure}
% \begin{itemize}
%     \item возьмем два вектора     
%     \begin{align*} 
%         w\quad &\mbox{-- яркости пикселей на template изображении} \\ 
%         g(x,y)\quad &\mbox{-- яркости в окне с центром (x,y) на исходном изображении $f$}
%     \end{align*}
%     \item косинус угла между векторами
%     $$
%         \cos\theta=\frac{w^t\cdot g(x,y)}{\norm{w}\cdot \norm{g(x,y)}}=
%         \frac{(w\bullet f)(x,y)}{\norm{w}\cdot \norm{g(x,y)}}
%     $$
%     \item template matching 
%     $$
%         x_\ast, y_\ast=\arg\max_{x,y}\cos\theta
%     $$ 
% \end{itemize}
% \end{frame}

%---------------------------------------------------------------------
\begin{frame}
    \frametitle{$\blacksquare$ Box--фильтр}
    \begin{center}
        \includegraphics[width=0.5\textwidth]{boxfilter.png}
    \end{center}
    
    \begin{itemize}
        \item box--фильтр размера $n\times n$
        $$
            g = w\circ f\,,\quad
            w = \frac{1}{n^2}
            \begin{bmatrix*}[r]
                1 & 1 & \ldots & 1\\
                1 & 1 & \ddots & 1\\
                1 & 1 & \ldots & 1
            \end{bmatrix*}
        $$
    \end{itemize}
\end{frame}

%---------------------------------------------------------------------
\begin{frame}
    \frametitle{Гауссовский фильтр}
    \begin{center}
        \includegraphics[width=0.6\textwidth]{gaussian1.pdf}
    \end{center}
    Сепарабельное ядро гауссовского фильтра размера $n\times n$
    $$
        w(s,t)=K\exp\left\{-\frac{s^2+t^2}{2\sigma^2}\right\}
        \qquad n=2k+1\qquad k\approx 3\sigma
    $$
    Гауссовский фильтр уменьшает шум, но размывает границы.
\end{frame}

%---------------------------------------------------------------------
\begin{frame}
    \frametitle{$\blacksquare$ Сравнение box и гауссовской фильтрации}
    \begin{center} 
        \includegraphics[width=0.7\textwidth]{gaussian2.pdf}
        \captionof*{figure}{Результат box и гауссовской фильтрации}
    \end{center}
\end{frame}

%---------------------------------------------------------------------
\begin{frame}
    \frametitle{$\blacksquare$ Медианная фильтрация}
    \begin{center}
        \includegraphics[width=0.5\textwidth]{median_filter.png}
    \end{center}
    \begin{itemize}
        \item нелинейный фильтр
        \item устойчив к выбросам (salt-and-pepper noise)
        \item не сильно размывает границы
    \end{itemize}
    \begin{center}
        \includegraphics[width=0.6\textwidth]{median.pdf}
    \end{center}
\end{frame}

%---------------------------------------------------------------------
% \begin{frame}
%     \frametitle{Фильтр Лапласа}
%     Оператор Лапласа
%     $$
%         \Delta f(x,y)=\frac{\partial^2 f}{\partial x^2}+\frac{\partial^2 f}{\partial y^2}
%     $$
%     \begin{center}
%         \includegraphics[width=0.4\textwidth]{laplacian.pdf}
%         \captionof*{figure}{5-точечный и 9-точечный  фильтр Лапласа}
%     \end{center}
%     Пятиточечная разностная схема оператора Лапласа
%     $$
%         \Delta f(x,y)\approx f(x+1,y)+f(x-1,y)+f(x,y+1)+f(x,y-1)-4f(x,y)
%     $$
% \end{frame}

%---------------------------------------------------------------------
% \begin{frame}
%     \frametitle{Вычисление частных производных изображения}
%     Градиент функции
%     $$
%         \nabla f(x,y) = [f'_x(x,y), f'_y(x,y)]
%     $$
%     Разностные схемы для частных производных
%     $$
%         f'_x \approx (z_3+2z_6+z_9)-(z_1+2z_4+z_7)
%     $$
%     $$
%         f'_y \approx (z_7+2z_8+z_9)-(z_1+2z_2+z_3)\,,\quad 
%     $$
%     \begin{figure}[t]%
%         \centering
%         \includegraphics[width=0.15\textwidth]{gradient2.pdf}
%         \quad
%         \includegraphics[width=0.3\textwidth]{gradient3.pdf}
%     \end{figure}
%     \begin{center}
%         \includegraphics[width=0.15\textwidth]{gradient1.pdf}
%     \end{center}
% \end{frame}

%---------------------------------------------------------------------
\begin{frame}
\frametitle{$\blacksquare$ Дискретный базис Фурье}
\begin{itemize}
    \item дискретный базис Фурье для изображения размера $H\times W$
    $$
        \varphi(x, y, u, v) = e^{-2\pi i(ux/W+vy/H)}
    $$
    \item координаты в пространственной области
    $$
        0\leqslant x< W\qquad
        0\leqslant y< H
    $$
    \item координаты в частотной области
    $$
        0\leqslant u< W\qquad
        0\leqslant v< H
    $$
    \begin{figure}
        \centering
        \includegraphics[height=0.35\textheight]{basis.png}
    \end{figure}
    $$
        \mathrm{Re}[\varphi]=\cos\left[2\pi(ux/W+vy/H)\right]
    $$
\end{itemize}
\end{frame}

%---------------------------------------------------------------------
\begin{frame}
\frametitle{Дискретное преобразование Фурье (DFT)}
\begin{figure}
    \centering
    \includegraphics[height=0.3\textheight]{dft.png}
\end{figure}
\begin{itemize}
    \item дискретное преобразование Фурье изображения $f(x,y)$ размера $H\times W$
    $$
        F(u, v)=\sum_{x=0}^{W-1}\sum_{y=0}^{H-1}
        f(x,y)\varphi(x,y,u,v)
    $$
    \item спектр Фурье
    $$
        |F(u, v)|
    $$
    \item фазовый спектр Фурье
    $$
        \arg F(u, v)
    $$
\end{itemize}
\end{frame}

%---------------------------------------------------------------------
\begin{frame}
    \frametitle{Сдвиг низких частот в центр}
    \begin{figure}
        \centering
        \includegraphics[height=0.32\textheight]{dft.png}
    \end{figure}
    Низкие частоты (крупные детали) -- по углам спектра. 
    Высокие частоты (мелкие детали) -- в середине спектра.
    \begin{figure}
        \centering
        \includegraphics[height=0.3\textheight]{fig4.png}
        \includegraphics[height=0.3\textheight]{fftshift.png}
        \includegraphics[height=0.3\textheight]{fig5.png}
    \end{figure}
    После сдвига: чем дальше от центра, тем выше частота.
\end{frame}

%---------------------------------------------------------------------
\begin{frame}
\frametitle{Быстрое преобразование Фурье (FFT)}
$$
    F(u, v)=\sum_{x=0}^{W-1}\sum_{y=0}^{H-1}f(x,y)\varphi(x,y,u,v)
$$    
$$
    0\leqslant u,x < W\qquad 
    0\leqslant v,y < H
$$
\begin{itemize}
    \item вычислительная сложность (число сложений и умножений) DFT
    $$
        \mathcal{O}[(HW)^2]
    $$ 
    \item вычислительная сложность быстрого преобразования Фурье
    $$
        HW\cdot \mathcal{O}[\ln(HW)]
    $$
\end{itemize}
\end{frame}

%---------------------------------------------------------------------
\begin{frame}
    \frametitle{Примеры спектров Фурье}
    \begin{figure}
        \centering
        \includegraphics[height=0.5\textheight]{DFT_examples_waves.png}
    \end{figure}
\end{frame}

%---------------------------------------------------------------------
\begin{frame}
    \frametitle{$\blacksquare$ Quiz}
    \begin{figure}
        \centering
        \includegraphics[height=0.7\textheight]{FTgame.png}
    \end{figure}
    \begin{itemize}
        \item<2-> 1-h, 2-f, 3-g, 4-c, 5-b, 6-e, 7-d, 8-a
    \end{itemize}
\end{frame}

%---------------------------------------------------------------------
\begin{frame}
\frametitle{Обратное дискретное преобразование Фурье (IDFT)}
\begin{itemize}
    \item дискретное преобразование Фурье
    $$
        F(u, v)=\sum_{x=0}^{W-1}\sum_{y=0}^{H-1}f(x,y)\varphi(x,y,u,v)
    $$  
    \item обратное дискретное преобразование Фурье
    $$
        f(x,y)=\frac{1}{HW}
        \sum_{u=0}^{W-1}\sum_{v=0}^{H-1}F(u,v)\varphi^\ast(x,y,u,v)
    $$
    \item обозначения
    $$
        F = \mathcal{F}[f]\,,\qquad
        f = \mathcal{F}^{-1}[F]
    $$
\end{itemize}
\end{frame}

%---------------------------------------------------------------------
\begin{frame}
    \frametitle{Примеры обратного преобразования Фурье}
    \begin{figure}
        \centering
        \includegraphics[height=0.8\textheight]{dft_random_phaseandamplitude.png}
    \end{figure}
\end{frame}

%---------------------------------------------------------------------
\begin{frame}
\frametitle{$\blacksquare$ Спектр Фурье при сдвиге и повороте изображения}
\begin{figure}
    \centering
    \includegraphics[height=0.5\textheight]{fig6.png}
    \includegraphics[height=0.5\textheight]{fig7.png}
\end{figure}
\begin{itemize}
    \item при сдвиге изображения спектр Фурье не меняется
    \item при повороте изображения на угол $\alpha$, спектр Фурье тоже повернется на угол $\alpha$
    \item фазовый спектр меняется в обоих случаях
\end{itemize}
\end{frame}

%---------------------------------------------------------------------
% \begin{frame}
% \frametitle{Теорема о свертке}
% \begin{itemize}
%     \item преобразование Фурье свертки равно произведению преобразований Фурье
%     $$
%         \mathcal{F}[w\circ f]=\mathcal{F}[w]\cdot \mathcal{F}[f]
%     $$
%     \item преобразование Фурье произведения двух изображений равно свертке их преобразований Фурье
%     $$
%         \mathcal{F}[f\cdot g]=\mathcal{F}[f]\circ \mathcal{F}[g]
%     $$
% \end{itemize}
% \end{frame}

%---------------------------------------------------------------------
% \begin{frame}
% \frametitle{Вычисление свертки с помощью преобразования Фурье}
% \begin{figure}
%     \centering
%     \includegraphics[width=0.3\textwidth]{gradient3.pdf}
%     \includegraphics[height=0.3\textheight]{fig9.pdf}
% \end{figure}
% \begin{itemize}
%     \item теорема о свертке
%     $$
%         \mathcal{F}[w\circ f]=\mathcal{F}[w]\cdot \mathcal{F}[f]
%     $$
%     \item вычисление свертки с помощью преобразования Фурье
%     $$
%         w\circ f =\mathcal{F}^{-1}[\mathcal{F}[w]\cdot \mathcal{F}[f]]
%     $$
%     \item вычислительная эффективность
%     $$
%         \frac{\mathcal{O}[HWn]}
%         {HW\cdot \mathcal{O}[\ln(HW)]}=
%         \mathcal{O}\left[\frac{n}{\ln(HW)}\right]
%     $$
% \end{itemize}
% \end{frame}

%---------------------------------------------------------------------
\begin{frame}
\frametitle{$\blacksquare$ Фильтрация изображений в частотной области}
\begin{figure}
    \centering
    \includegraphics[height=0.5\textheight]{fig11.pdf}
    \includegraphics[height=0.5\textheight]{fig12.pdf}
\end{figure}
\begin{itemize}
    \item фильтрация шума в частотной области
    $$
        M\,\,\mbox{-- маска в частотной области}\,,\quad F=\mathcal{F}[f]
        \qquad\mapsto\qquad 
        g =\mathcal{F}^{-1}[M\cdot F]
    $$
\end{itemize}
\end{frame}

%--------------------------------------------------------
\end{document}
