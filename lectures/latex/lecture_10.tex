\documentclass[12pt, usepdftitle=false, aspectratio=1610]{beamer}

\usetheme{Madrid}

\usefonttheme[]{serif}
\setbeamertemplate{caption}[numbered]
\setbeamertemplate{navigation symbols}{}

\usepackage[T2A]{fontenc}			
\usepackage[utf8]{inputenc}			
\usepackage[english,russian]{babel}

\usepackage{
    mathtext,
    cmap,
    multirow,
    textcomp,
    graphicx,
    wrapfig,
    subfig,
    mathtools,
    gensymb,
    amsmath,
    hyperref,
    verbatim
}
\usepackage[font=small,labelfont=bf]{caption}
\usepackage[absolute, overlay]{textpos}

\DeclarePairedDelimiter{\norm}{\lVert}{\rVert} 
\DeclareMathOperator*{\argmax}{arg\,max}
\DeclareMathOperator*{\argmin}{arg\,min}

\graphicspath{{./figs/10}}

\title[Лекция 10]{
    PnP задача для сцен на плоскости. \\
    Калибровка камеры.
}

\author{Александр Танченко}
\institute{}
\date{2025}

\begin{document}

%------------------------------------------------------------
\begin{frame}
\titlepage
\end{frame}

%------------------------------------------------------------
\begin{frame}
\begin{figure}[h]
    \centering
    \includegraphics[height=0.4\textheight]{scene3d.jpg}
\end{figure}
Для трехмерных сцен, преобразование из всемирной системы координат в систему координат изображения
задается pinhole моделью камеры:
$$
    \lambda\cdot
    \begin{bmatrix}
        x \\ y \\ 1 
    \end{bmatrix} =
    \mathbf{P} \cdot
    \begin{bmatrix}
        X \\ Y \\ Z \\ 1 
    \end{bmatrix}
$$
где матрица камеры
$$
    \mathbf{P} =
        \mathbf{K} \cdot
        \begin{bmatrix}
            \mathbf{R} \mid \mathbf{t}
        \end{bmatrix}
$$
\end{frame}

%------------------------------------------------------------
\begin{frame}
    \begin{figure}[h]
    \centering
    \includegraphics[height=0.4\textheight]{homography.pdf}
\end{figure}
\begin{itemize}
    \item Прямым на одном изображении соответствуют прямые на другом.
    
    \item Значит координаты соответствующих точек на изображениях связаны гомографией.
\end{itemize}
\end{frame}

%------------------------------------------------------------
\begin{frame}
    \begin{figure}[h]
    \centering
    \includegraphics[height=0.4\textheight]{scene2d.jpg}
\end{figure}
\begin{itemize}
    \item Для сцен на плоскости, прямые на плоскости сцены проектируются в прямые на плоскости изображения.
    \item Значит преобразование координат между плоскостью сцены и плоскостью изображения задается гомографией.
\end{itemize}
\end{frame}

%------------------------------------------------------------
\begin{frame}
    \begin{figure}[h]
    \centering
    \includegraphics[height=0.35\textheight]{plane.pdf}
\end{figure}
\hspace{1cm}

Гомографию можно найти, взяв $Z=0$ в модели камеры:
$$
    \lambda\cdot
    \begin{bmatrix}
        x \\ y \\ 1 
    \end{bmatrix} =
    \mathbf{P} \cdot
    \begin{bmatrix}
        X \\ Y \\ 0 \\ 1 
    \end{bmatrix}
    \qquad\Rightarrow\qquad
    \lambda\cdot
    \begin{bmatrix}
        x \\ y \\ 1 
    \end{bmatrix} =
    \mathbf{H} \cdot
    \begin{bmatrix}
        X \\ Y \\ 1 
    \end{bmatrix}
$$
\end{frame}

%------------------------------------------------------------
\begin{frame}
    \begin{figure}[h]
    \centering
    \includegraphics[height=0.4\textheight]{pnp.jpg}
\end{figure}

Основные задачи 3D CV для сцен на плоскости остаются такими же как и для трехмерных сцен.
\vspace{0.5cm}

\textbf{PnP problem: оценка внешних параметров камеры} 
\vspace{0.5cm}

  По точкам на плоской сцене $(X_i, Y_i)$ и их проекциям на изображение камеры $(x_i,y_i)$ 
  требуется оценить внешние параметры камеры $\mathbf{R}$, $\mathbf{t}$, 
  при этом внутренние параметры камеры $\mathbf{K}$ считаются известными. 
\end{frame}

%------------------------------------------------------------
\begin{frame}
    \begin{figure}[h]
    \centering
    \includegraphics[height=0.4\textheight]{triangulation.pdf}
\end{figure}

\textbf{Triangulation problem: реконструкция точек сцены} 
\vspace{0.5cm}

Зная внешние и внутренние параметры камер $\mathbf{K}_j$, $\mathbf{R}_j$, $\mathbf{t}_j$ 
и координаты точек на изображениях $(x_{ij}, y_{ij})$, 
требуется оценить координаты точек на плоскости сцены $(X_i, Y_i)$.
\end{frame}

%------------------------------------------------------------
\begin{frame}
\begin{figure}[h]
    \centering
    \includegraphics[height=0.4\textheight]{calibration.jpg}
\end{figure}

\textbf{Camera calibration problem: калибровка камеры} 
\vspace{0.5cm}

Зная координаты точек на плоских шаблонах $(X_i, Y_i)$ 
и их проекции на нескольких изображениях с одной камеры 
$(x_{ij},y_{ij})$, требуется оценить внутренние параметры камеры 
$\mathbf{K}$ и коэффициенты дисторсии
$$
    k_1\,,k_2\,,k_3\,,p_1\,,p_2
$$
\end{frame}

%------------------------------------------------------------
\begin{frame}
\begin{figure}[h]
    \centering
    \includegraphics[height=0.4\textheight]{chess.jpeg}
    \includegraphics[height=0.45\textheight]{aruco.png}
\end{figure}

\textbf{Для калибровки можно использовать:}

\begin{itemize}
    \item \href{https://markhedleyjones.com/projects/calibration-checkerboard-collection}
        {Checkerboard patterns}
    
    \item \href{https://chev.me/arucogen/}{AruCo patterns} 
\end{itemize}
\hspace{0.5cm}

Для калибровки необходимо как минимум 3 изображения калибровочного паттерна под разными ракурсами.

Обычно используют $\sim 30$ изображений паттерна.
\end{frame}

%------------------------------------------------------------
\begin{frame}
\begin{figure}[h]
    \centering
    \includegraphics[height=0.4\textheight]{distortion.jpg}
    \hspace{1cm}
    \includegraphics[height=0.45\textheight]{fisheye_image.jpg}
\end{figure}
\begin{itemize}
    \item Дисторсия возникает из-за несовершенства оптических систем камер.
    
    \item Для fisheye-камер дисторсия делается для увеличения поля зрения камеры 
    (FOV – Field of View).
\end{itemize}
\end{frame}

%------------------------------------------------------------
\begin{frame}
    Дисторсия координат происходит перед умножением на матрицу $\mathbf{K}$:
    $$
        \lambda\cdot
        \begin{bmatrix}
            x \\ y \\ 1 
        \end{bmatrix} =
        \mathbf{K} \cdot
        \begin{bmatrix}
            \mathbf{R} \mid \mathbf{t}
        \end{bmatrix}
        \cdot
        \begin{bmatrix}
            X \\ Y \\ Z \\ 1 
        \end{bmatrix}
    $$
    \begin{itemize}
    \item из мировой системы в систему координат камеры:
    $$
        \begin{bmatrix}
            X \\ Y \\ Z 
        \end{bmatrix}
        \quad\rightarrow\quad
        \begin{bmatrix}
            X_c \\ Y_c \\ Z_c 
        \end{bmatrix} = 
        \mathbf{R}\cdot
        \begin{bmatrix}
            X \\ Y \\ Z 
        \end{bmatrix} +
        \mathbf{t}
        \quad\rightarrow
    $$
    \item переход к нормализованным координатам:
    $$
        \begin{bmatrix}
            X_c \\ Y_c \\ Z_c 
        \end{bmatrix}
        \quad\rightarrow\quad
        \begin{bmatrix}
            x \\ y 
        \end{bmatrix} = 
        \begin{bmatrix}
            X_c/Z_c \\ Y_c/Z_c 
        \end{bmatrix}
        \quad\rightarrow
    $$
    \end{itemize}
\end{frame}

%------------------------------------------------------------
\begin{frame}
\begin{itemize}
    \item дисторсия координат (добавление \textbf{карт ректификации} $m_x$, $m_y$):
    $$
        \begin{bmatrix}
            x \\ y 
        \end{bmatrix}
        \quad\rightarrow\quad
        \begin{bmatrix}
            x_d \\ y_d 
        \end{bmatrix} = 
        \begin{bmatrix}
            x + m_x \\ y + m_y 
        \end{bmatrix}
        \quad\rightarrow
    $$
    \item переход к координатам пикселей на изображении
    $$
        \begin{bmatrix}
            x_d \\ y_d 
        \end{bmatrix}
        \quad\rightarrow\quad
        \begin{bmatrix}
            x \\ y \\ 1 
        \end{bmatrix} = 
        \mathbf{K}\cdot
        \begin{bmatrix}
            x_d \\ y_d \\ 1 
        \end{bmatrix}
    $$
\end{itemize}
\end{frame}

%------------------------------------------------------------
\begin{frame}
\begin{itemize}
    \item \textbf{Модель радиальной дисторсии}
    $$
        m_x = x_{d} - x = x \cdot (k_1 r^2 + k_2 r^4 + k_3 r^6) 
    $$
    где
    $$
        r = \sqrt{(x - c_x)^2 + (y - c_y)^2}
    $$

    \item \textbf{Модель тангенциальной дисторсии}
    $$
        m_x = 2 p_1 x y + p_2 (r^2 + 2 x^2)
    $$
    $$
        m_y = p_1 (r^2 + 2 y^2) + 2 p_2 x y
    $$

    \item Для fisheye-камеры есть своя модель, которую мы не рассматриваем.
\end{itemize}
\end{frame}

\begin{frame}
    \textbf{Замечания к трём задачам:}
    \hspace{1cm}
    
    \begin{itemize}
        \item Все три задачи решаются методами нелинейной оптимизации.
        
        \item Для получения начального приближения используются методы DLT (Direct Linear Transform).
        
        \item DLT алгоритмы используются для аугментации изображений в DL/CV.
    \end{itemize}
\end{frame}

%------------------------------------------------------------
\begin{frame}[fragile]
    \begin{figure}[h]
    \centering
    \includegraphics[height=0.35\textheight]{calib.jpg}
\end{figure}
\textbf{Функции OpenCV для калибровки камеры}
\begin{itemize}
    \item Поиск угловых точек на калибровочном паттерне
    \begin{verbatim}
    _, points = cv2.findChessboardCorners(gray_image, patternSize)
    \end{verbatim}
    \item Калибровка камеры
    \begin{verbatim}
    _, K, dist_coeff, rvecs, tvecs = cv2.calibrateCamera(
    objectPoints,  # список координат угловых точек в 3D 
    imagePoints,   # список координат угловых точек на изображениях
    (W, H),        # размер изображения
    ...
)
    \end{verbatim}
\end{itemize}
\end{frame}

%------------------------------------------------------------
\begin{frame}[fragile]
\textbf{Функции OpenCV для коррекции дисторсии}
\begin{itemize}
    \item Построение карт ректификации
    \begin{verbatim}
    mx, my = cv2.initUndistortRectifyMap(
        K, 
        dist_coeff, 
        None, None, 
        (image_width, image_height), 
        cv2.CV_32FC1
    )
    \end{verbatim}
    \item Коррекция дисторсии
    \begin{verbatim}
    rectified_image = cv2.remap(
        distorted_image, 
        mx, my, 
        interpolation=cv2.INTER_LINEAR
    )
    \end{verbatim}
\end{itemize}
\end{frame}

%------------------------------------------------------------
\begin{frame}[fragile]
\textbf{Функции OpenCV для PnP задачи}
\begin{itemize}
\item Фильтрация точек с помощью алгоритма RANSAC
\begin{verbatim}
_, rvec, tvec, inliers = cv2.solvePnPRansac(
    objectPoints, # координаты точек в 3D
    imagePoints,  # координаты точек на изображении
    K,            # camera intrinsics
    dist_coeff,   # коэффициенты дисторсии или None
    ...
)
\end{verbatim}
\item Оценка camera extrinsics с помощью нелинейной оптимизации
\begin{verbatim}
_, rvec, tvec = cv2.solvePnP(
    objectPoints, 
    imagePoints, 
    K, 
    dist_coeff
)
\end{verbatim}
\end{itemize}

\end{frame}

%------------------------------------------------------------
\end{document}

        